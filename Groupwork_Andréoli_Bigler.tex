% Options for packages loaded elsewhere
\PassOptionsToPackage{unicode}{hyperref}
\PassOptionsToPackage{hyphens}{url}
%
\documentclass[
]{article}
\usepackage{amsmath,amssymb}
\usepackage{lmodern}
\usepackage{ifxetex,ifluatex}
\ifnum 0\ifxetex 1\fi\ifluatex 1\fi=0 % if pdftex
  \usepackage[T1]{fontenc}
  \usepackage[utf8]{inputenc}
  \usepackage{textcomp} % provide euro and other symbols
\else % if luatex or xetex
  \usepackage{unicode-math}
  \defaultfontfeatures{Scale=MatchLowercase}
  \defaultfontfeatures[\rmfamily]{Ligatures=TeX,Scale=1}
\fi
% Use upquote if available, for straight quotes in verbatim environments
\IfFileExists{upquote.sty}{\usepackage{upquote}}{}
\IfFileExists{microtype.sty}{% use microtype if available
  \usepackage[]{microtype}
  \UseMicrotypeSet[protrusion]{basicmath} % disable protrusion for tt fonts
}{}
\makeatletter
\@ifundefined{KOMAClassName}{% if non-KOMA class
  \IfFileExists{parskip.sty}{%
    \usepackage{parskip}
  }{% else
    \setlength{\parindent}{0pt}
    \setlength{\parskip}{6pt plus 2pt minus 1pt}}
}{% if KOMA class
  \KOMAoptions{parskip=half}}
\makeatother
\usepackage{xcolor}
\IfFileExists{xurl.sty}{\usepackage{xurl}}{} % add URL line breaks if available
\IfFileExists{bookmark.sty}{\usepackage{bookmark}}{\usepackage{hyperref}}
\hypersetup{
  pdftitle={Groupwork R Bootcamp},
  pdfauthor={Felix Bigler, Régis Andréoli},
  hidelinks,
  pdfcreator={LaTeX via pandoc}}
\urlstyle{same} % disable monospaced font for URLs
\usepackage[margin=1in]{geometry}
\usepackage{color}
\usepackage{fancyvrb}
\newcommand{\VerbBar}{|}
\newcommand{\VERB}{\Verb[commandchars=\\\{\}]}
\DefineVerbatimEnvironment{Highlighting}{Verbatim}{commandchars=\\\{\}}
% Add ',fontsize=\small' for more characters per line
\usepackage{framed}
\definecolor{shadecolor}{RGB}{248,248,248}
\newenvironment{Shaded}{\begin{snugshade}}{\end{snugshade}}
\newcommand{\AlertTok}[1]{\textcolor[rgb]{0.94,0.16,0.16}{#1}}
\newcommand{\AnnotationTok}[1]{\textcolor[rgb]{0.56,0.35,0.01}{\textbf{\textit{#1}}}}
\newcommand{\AttributeTok}[1]{\textcolor[rgb]{0.77,0.63,0.00}{#1}}
\newcommand{\BaseNTok}[1]{\textcolor[rgb]{0.00,0.00,0.81}{#1}}
\newcommand{\BuiltInTok}[1]{#1}
\newcommand{\CharTok}[1]{\textcolor[rgb]{0.31,0.60,0.02}{#1}}
\newcommand{\CommentTok}[1]{\textcolor[rgb]{0.56,0.35,0.01}{\textit{#1}}}
\newcommand{\CommentVarTok}[1]{\textcolor[rgb]{0.56,0.35,0.01}{\textbf{\textit{#1}}}}
\newcommand{\ConstantTok}[1]{\textcolor[rgb]{0.00,0.00,0.00}{#1}}
\newcommand{\ControlFlowTok}[1]{\textcolor[rgb]{0.13,0.29,0.53}{\textbf{#1}}}
\newcommand{\DataTypeTok}[1]{\textcolor[rgb]{0.13,0.29,0.53}{#1}}
\newcommand{\DecValTok}[1]{\textcolor[rgb]{0.00,0.00,0.81}{#1}}
\newcommand{\DocumentationTok}[1]{\textcolor[rgb]{0.56,0.35,0.01}{\textbf{\textit{#1}}}}
\newcommand{\ErrorTok}[1]{\textcolor[rgb]{0.64,0.00,0.00}{\textbf{#1}}}
\newcommand{\ExtensionTok}[1]{#1}
\newcommand{\FloatTok}[1]{\textcolor[rgb]{0.00,0.00,0.81}{#1}}
\newcommand{\FunctionTok}[1]{\textcolor[rgb]{0.00,0.00,0.00}{#1}}
\newcommand{\ImportTok}[1]{#1}
\newcommand{\InformationTok}[1]{\textcolor[rgb]{0.56,0.35,0.01}{\textbf{\textit{#1}}}}
\newcommand{\KeywordTok}[1]{\textcolor[rgb]{0.13,0.29,0.53}{\textbf{#1}}}
\newcommand{\NormalTok}[1]{#1}
\newcommand{\OperatorTok}[1]{\textcolor[rgb]{0.81,0.36,0.00}{\textbf{#1}}}
\newcommand{\OtherTok}[1]{\textcolor[rgb]{0.56,0.35,0.01}{#1}}
\newcommand{\PreprocessorTok}[1]{\textcolor[rgb]{0.56,0.35,0.01}{\textit{#1}}}
\newcommand{\RegionMarkerTok}[1]{#1}
\newcommand{\SpecialCharTok}[1]{\textcolor[rgb]{0.00,0.00,0.00}{#1}}
\newcommand{\SpecialStringTok}[1]{\textcolor[rgb]{0.31,0.60,0.02}{#1}}
\newcommand{\StringTok}[1]{\textcolor[rgb]{0.31,0.60,0.02}{#1}}
\newcommand{\VariableTok}[1]{\textcolor[rgb]{0.00,0.00,0.00}{#1}}
\newcommand{\VerbatimStringTok}[1]{\textcolor[rgb]{0.31,0.60,0.02}{#1}}
\newcommand{\WarningTok}[1]{\textcolor[rgb]{0.56,0.35,0.01}{\textbf{\textit{#1}}}}
\usepackage{graphicx}
\makeatletter
\def\maxwidth{\ifdim\Gin@nat@width>\linewidth\linewidth\else\Gin@nat@width\fi}
\def\maxheight{\ifdim\Gin@nat@height>\textheight\textheight\else\Gin@nat@height\fi}
\makeatother
% Scale images if necessary, so that they will not overflow the page
% margins by default, and it is still possible to overwrite the defaults
% using explicit options in \includegraphics[width, height, ...]{}
\setkeys{Gin}{width=\maxwidth,height=\maxheight,keepaspectratio}
% Set default figure placement to htbp
\makeatletter
\def\fps@figure{htbp}
\makeatother
\setlength{\emergencystretch}{3em} % prevent overfull lines
\providecommand{\tightlist}{%
  \setlength{\itemsep}{0pt}\setlength{\parskip}{0pt}}
\setcounter{secnumdepth}{5}
\ifluatex
  \usepackage{selnolig}  % disable illegal ligatures
\fi

\title{Groupwork R Bootcamp}
\author{Felix Bigler, Régis Andréoli}
\date{2/23/2022}

\begin{document}
\maketitle

{
\setcounter{tocdepth}{4}
\tableofcontents
}
\begin{verbatim}
## Warning: package 'dplyr' was built under R version 4.0.5
\end{verbatim}

\begin{verbatim}
## Warning: package 'nlme' was built under R version 4.0.5
\end{verbatim}

\newpage

\hypertarget{intrudction}{%
\section{Intrudction}\label{intrudction}}

In this group work for the R bootcamp module, we would like to analyze a
photovoltaic system dataset. The dataset was created during the
operation of AEW-owned PV plants in 2019. In total there are 3 different
plants but we will only concentrate on one (Plant A). The plants are all
located in Aargau, Switzerland. The power values in kW refer to the
average over the 15min period. The data was published for the Energy
Data Hackdays 2020 in Brugg.

We then merge this dataset with weather data from 2019 in order to
evaluate the influence of the weather as well as be able to build a
simple prediction model. Moreover we would like to create a simple
battery charging algorithm in order to store surplus energy.

\hypertarget{data-preperation}{%
\section{Data Preperation}\label{data-preperation}}

\hypertarget{load-data}{%
\subsection{Load Data}\label{load-data}}

\begin{Shaded}
\begin{Highlighting}[]
\CommentTok{\# set directory read data}
\FunctionTok{setwd}\NormalTok{(}\FunctionTok{dirname}\NormalTok{(}\FunctionTok{getActiveDocumentContext}\NormalTok{()}\SpecialCharTok{$}\NormalTok{path))}
\NormalTok{df\_weather }\OtherTok{\textless{}{-}} \FunctionTok{read.csv}\NormalTok{(}\StringTok{"./data/weather.csv"}\NormalTok{,}\AttributeTok{header=}\ConstantTok{TRUE}\NormalTok{,}\AttributeTok{sep =}\StringTok{","}\NormalTok{,}\AttributeTok{comment.char =}\StringTok{"\#"}\NormalTok{)}
\NormalTok{df\_plantA }\OtherTok{\textless{}{-}} \FunctionTok{read.csv}\NormalTok{(}\StringTok{"./data/A.csv"}\NormalTok{,}\AttributeTok{header=}\ConstantTok{TRUE}\NormalTok{,}\AttributeTok{sep =}\StringTok{","}\NormalTok{)}
\end{Highlighting}
\end{Shaded}

\hypertarget{inspect-data-structure}{%
\subsection{Inspect Data Structure}\label{inspect-data-structure}}

\begin{Shaded}
\begin{Highlighting}[]
\FunctionTok{str}\NormalTok{(df\_weather)}
\FunctionTok{str}\NormalTok{(df\_plantA)}
\end{Highlighting}
\end{Shaded}

We got 8760 observations for the weather dataset and 35040 observations
for each PV plant Dataset. Which makes sense since we got weather data
for every hour for the year 2019 and on the other hand, operational data
of the PV plants for every 15 min in kW. The values are all of the type
num except the datetime entries. These are of type char. In a next step
we convert the datetime entries to a datetime object to have the
possibility to work with the datetime entries. There are no missing
values so we don't have to

\hypertarget{convert-datetime-entries-from-char-to-datetime-object}{%
\subsection{Convert datetime entries from char to datetime
object}\label{convert-datetime-entries-from-char-to-datetime-object}}

Since all datetime are from the type char we convert the local time to a
datetime object.

\begin{Shaded}
\begin{Highlighting}[]
\NormalTok{df\_weather}\SpecialCharTok{$}\NormalTok{local\_time }\OtherTok{\textless{}{-}} \FunctionTok{as.POSIXct}\NormalTok{(df\_weather}\SpecialCharTok{$}\NormalTok{local\_time,}\AttributeTok{tz=}\StringTok{"GMT"}\NormalTok{,}\AttributeTok{format=}\StringTok{"\%Y{-}\%m{-}\%d \%H:\%M"}\NormalTok{)}
\NormalTok{df\_plantA}\SpecialCharTok{$}\NormalTok{Timestamp }\OtherTok{\textless{}{-}} \FunctionTok{as.POSIXct}\NormalTok{(df\_plantA}\SpecialCharTok{$}\NormalTok{Timestamp,}\AttributeTok{tz=}\StringTok{"GMT"}\NormalTok{,}\AttributeTok{format=}\StringTok{"\%Y{-}\%m{-}\%d \%H:\%M:\%S"}\NormalTok{)}
\end{Highlighting}
\end{Shaded}

\hypertarget{resample-pv-plant-timeseries-from-15min-to-hourly-intervalls}{%
\subsection{Resample PV plant Timeseries from 15min to Hourly
Intervalls}\label{resample-pv-plant-timeseries-from-15min-to-hourly-intervalls}}

Since the PV plant entries are 15 minute observations and the weather
data is hourly we resample the PV plant dataset by grouping every 15 min
observation to its corresponding hour and take the sum of it. We got
then directly the energy gained in kWh instead of the 15 min PV power in
kW which is more convenient.

\begin{Shaded}
\begin{Highlighting}[]
\NormalTok{df\_plantA\_resample }\OtherTok{\textless{}{-}}\NormalTok{ df\_plantA }\SpecialCharTok{\%\textgreater{}\%}
  \FunctionTok{mutate}\NormalTok{(}\AttributeTok{datetime =} \FunctionTok{floor\_date}\NormalTok{(Timestamp, }\StringTok{"1 hour"}\NormalTok{)) }\SpecialCharTok{\%\textgreater{}\%}
  \FunctionTok{group\_by}\NormalTok{(datetime) }\SpecialCharTok{\%\textgreater{}\%}
  \FunctionTok{summarise}\NormalTok{(}\FunctionTok{across}\NormalTok{(Generation\_kW}\SpecialCharTok{:}\NormalTok{Overall\_Consumption\_Calc\_kW, sum)) }
\end{Highlighting}
\end{Shaded}

\hypertarget{merge-datasets}{%
\subsection{Merge Datasets}\label{merge-datasets}}

In this step we merge both datasets with an inner\_join by the entry
datetime. For that we have to rename the column local\_time to datetime
in the weather dataset.

\begin{Shaded}
\begin{Highlighting}[]
\NormalTok{df\_weather }\OtherTok{\textless{}{-}}\NormalTok{ df\_weather }\SpecialCharTok{\%\textgreater{}\%} 
  \FunctionTok{rename}\NormalTok{(}
    \AttributeTok{datetime =}\NormalTok{ local\_time,}
\NormalTok{    )}
\end{Highlighting}
\end{Shaded}

\begin{Shaded}
\begin{Highlighting}[]
\NormalTok{df\_A\_joined }\OtherTok{\textless{}{-}} \FunctionTok{inner\_join}\NormalTok{(df\_plantA\_resample, df\_weather, }\AttributeTok{by =} \StringTok{"datetime"}\NormalTok{)}
\end{Highlighting}
\end{Shaded}

\hypertarget{create-categorical-variable}{%
\subsection{Create Categorical
Variable}\label{create-categorical-variable}}

A prerequisite of the course assignment is that the data set contains at
least one categorical variable. Since our data set does not have any
categorical variable we create two of them by extracting the month as
well as the day out of the date time entries with the package lubridate
as follows.

\begin{Shaded}
\begin{Highlighting}[]
\NormalTok{df\_A\_joined }\OtherTok{\textless{}{-}}\NormalTok{df\_A\_joined }\SpecialCharTok{\%\textgreater{}\%}
 \FunctionTok{mutate}\NormalTok{(}
  \AttributeTok{month =} \FunctionTok{month}\NormalTok{(datetime),}
  \AttributeTok{month\_label =} \FunctionTok{month}\NormalTok{(datetime, }\AttributeTok{label=}\ConstantTok{TRUE}\NormalTok{),}
  \AttributeTok{hour =} \FunctionTok{hour}\NormalTok{(datetime),}
  \AttributeTok{day =} \FunctionTok{day}\NormalTok{(datetime),}
  \AttributeTok{year =} \FunctionTok{year}\NormalTok{(datetime)}
\NormalTok{ )}
\end{Highlighting}
\end{Shaded}

\hypertarget{graphical-observations-of-datasets}{%
\section{Graphical Observations of
Datasets}\label{graphical-observations-of-datasets}}

In this chapter we would like to explore the dataset graphically. First
we are going create a correlation matrix in order to see which variables
correlate with the PV plant production rate. Then we explore the
continuous as well as the categorical variables graphically.

\hypertarget{create-correlation-matrix}{%
\subsection{Create Correlation Matrix}\label{create-correlation-matrix}}

\begin{Shaded}
\begin{Highlighting}[]
\NormalTok{knitr}\SpecialCharTok{::}\NormalTok{opts\_chunk}\SpecialCharTok{$}\FunctionTok{set}\NormalTok{(}\AttributeTok{fig.width=}\DecValTok{12}\NormalTok{, }\AttributeTok{fig.height=}\DecValTok{8}\NormalTok{)}
\NormalTok{df\_corr }\OtherTok{\textless{}{-}} \FunctionTok{select}\NormalTok{(df\_A\_joined,}\SpecialCharTok{{-}}\FunctionTok{c}\NormalTok{(datetime,time,month\_label,year,day,month)) }
\CommentTok{\# Exclude datetime entries in order to create correlation matrix}
\NormalTok{corr }\OtherTok{\textless{}{-}} \FunctionTok{round}\NormalTok{(}\FunctionTok{cor}\NormalTok{(df\_corr), }\DecValTok{1}\NormalTok{)}
\CommentTok{\#head(corr[, 1:6])}
\end{Highlighting}
\end{Shaded}

\begin{Shaded}
\begin{Highlighting}[]
\FunctionTok{head}\NormalTok{(df\_corr)}
\end{Highlighting}
\end{Shaded}

\begin{Shaded}
\begin{Highlighting}[]
\NormalTok{p1 }\OtherTok{\textless{}{-}} \FunctionTok{ggcorrplot}\NormalTok{(corr, }\AttributeTok{hc.order =} \ConstantTok{TRUE}\NormalTok{, }\AttributeTok{type =} \StringTok{"lower"}\NormalTok{,}
   \AttributeTok{outline.col =} \StringTok{"white"}\NormalTok{,}
   \AttributeTok{ggtheme =}\NormalTok{ ggplot2}\SpecialCharTok{::}\NormalTok{theme\_gray,}
   \AttributeTok{lab =} \ConstantTok{TRUE}\NormalTok{,}
   \AttributeTok{colors =} \FunctionTok{c}\NormalTok{(}\StringTok{"\#6D9EC1"}\NormalTok{, }\StringTok{"white"}\NormalTok{, }\StringTok{"\#E46726"}\NormalTok{)) }\SpecialCharTok{+} 
   \FunctionTok{labs}\NormalTok{(}\AttributeTok{title =} \StringTok{"Correlation Matrix"}\NormalTok{,}
       \AttributeTok{subtitle =} \StringTok{"PV Plant Production vs. Weather Conditions"}\NormalTok{)}
\NormalTok{p1}
\end{Highlighting}
\end{Shaded}

\includegraphics{Groupwork_Andréoli_Bigler_files/figure-latex/unnamed-chunk-13-1.pdf}

We can conclude from the correlation matrix, that the variables
radiation\_surface (Ground-level solar irradiance (W / m²)) and
radiation\_toa (Top of atmosphere solar irradiance (W / m²)) have a
strong positive correlation with the PV production rate (variable
Generation\_kW). The coefficients are very high with a value of 0.9
which seems obvious, since a solar cell converts solar radiation into
electrical energy when sunlight is present on the surface of the cell.
The two variables certainly have a strong colinearity since the values
differ only in respect to the fact that one value was measured at the
ground and the other value above the atmosphere. For the actual solar
production, however, only the value at the ground is relevant.

The variable temperature has also a rather high correlation with a value
of 0.6. However, this correlation may have more in common with the fact
that the temperatures are also higher in the case of strong solar
irradiation. In fact, solar modules have a negative temperature
coefficient in relation to the efficiency rating
(\url{https://www.pveurope.eu/solar-modules/global-warming-growing-importance-temperature-coefficient-solar-modules}).

The other weather variables don't have a significant influence,
therefore we concentrate on the variable irradiance in the further
analysis of the data set.

\newpage

\hypertarget{exploration-of-contineous-variables}{%
\subsection{Exploration of Contineous
Variables}\label{exploration-of-contineous-variables}}

\hypertarget{yearly-ground-level-irradiance}{%
\subsubsection{Yearly Ground-Level
Irradiance}\label{yearly-ground-level-irradiance}}

\begin{Shaded}
\begin{Highlighting}[]
\NormalTok{p2 }\OtherTok{\textless{}{-}}  \FunctionTok{ggplot}\NormalTok{(}\AttributeTok{data =}\NormalTok{ df\_A\_joined,}
         \AttributeTok{mapping =} \FunctionTok{aes}\NormalTok{(}\AttributeTok{y =}\NormalTok{ Generation\_kW ,}
                     \AttributeTok{x =}\NormalTok{ radiation\_surface)) }\SpecialCharTok{+} 
  \FunctionTok{xlab}\NormalTok{(}\StringTok{"Ground{-}level solar irradiance [W / m²]"}\NormalTok{) }\SpecialCharTok{+} 
  \FunctionTok{ylab}\NormalTok{(}\StringTok{"PV generation [kWh]"}\NormalTok{) }\SpecialCharTok{+} 
     \FunctionTok{geom\_point}\NormalTok{(}\AttributeTok{alpha =} \FloatTok{0.5}\NormalTok{,}\AttributeTok{shape =} \DecValTok{1}\NormalTok{) }\SpecialCharTok{+}
     \FunctionTok{geom\_smooth}\NormalTok{(}\AttributeTok{method =} \StringTok{"lm"}\NormalTok{) }

\NormalTok{p3 }\OtherTok{\textless{}{-}}  \FunctionTok{ggplot}\NormalTok{(}\AttributeTok{data =}\NormalTok{ df\_A\_joined,}
         \AttributeTok{mapping =} \FunctionTok{aes}\NormalTok{(}\AttributeTok{y =}\NormalTok{ Generation\_kW ,}
                     \AttributeTok{x =}\NormalTok{ datetime)) }\SpecialCharTok{+} \FunctionTok{xlab}\NormalTok{(}\StringTok{"Date"}\NormalTok{) }\SpecialCharTok{+} \FunctionTok{ylab}\NormalTok{(}\StringTok{"PV generation [kWh]"}\NormalTok{) }\SpecialCharTok{+} 
     \FunctionTok{geom\_point}\NormalTok{(}\AttributeTok{alpha =} \FloatTok{0.2}\NormalTok{,}\AttributeTok{shape =} \DecValTok{1}\NormalTok{) }\SpecialCharTok{+}
     \FunctionTok{labs}\NormalTok{(}\AttributeTok{colour =} \StringTok{"Method"}\NormalTok{)}

\FunctionTok{grid.arrange}\NormalTok{(p2,p3, }\AttributeTok{nrow=}\DecValTok{2}\NormalTok{,}\AttributeTok{top =} \StringTok{"Yearly Ground{-}Level Irradiance"}\NormalTok{)}
\end{Highlighting}
\end{Shaded}

\begin{verbatim}
## `geom_smooth()` using formula 'y ~ x'
\end{verbatim}

\includegraphics{Groupwork_Andréoli_Bigler_files/figure-latex/unnamed-chunk-14-1.pdf}
* we can conclude from the plot above, that the PV production highly
correlates with the gorund level solar irradiance as the correlation
plot already explained.The blue line indicates a linear regression line.

\hypertarget{exploration-of-categorical-variables}{%
\subsection{Exploration of Categorical
Variables}\label{exploration-of-categorical-variables}}

\hypertarget{seasonal-and-timely-ground-level-irradiance}{%
\subsubsection{Seasonal and Timely Ground-Level
Irradiance}\label{seasonal-and-timely-ground-level-irradiance}}

\includegraphics{Groupwork_Andréoli_Bigler_files/figure-latex/unnamed-chunk-15-1.pdf}

\begin{itemize}
\tightlist
\item
  as we have expected, the most solar irradiance occurs between April
  and September and during the hours from 10am to 3pm as the boxplots
  above indicates.
\end{itemize}

\newpage

\hypertarget{pv-plant-production}{%
\subsection{PV Plant Production}\label{pv-plant-production}}

\begin{Shaded}
\begin{Highlighting}[]
\FunctionTok{library}\NormalTok{(viridis)}
\end{Highlighting}
\end{Shaded}

\begin{verbatim}
## Loading required package: viridisLite
\end{verbatim}

\begin{Shaded}
\begin{Highlighting}[]
\FunctionTok{library}\NormalTok{(ggExtra)}
\NormalTok{p6 }\OtherTok{\textless{}{-}} \FunctionTok{ggplot}\NormalTok{(}\AttributeTok{data =}\NormalTok{ df\_A\_joined, }\FunctionTok{aes}\NormalTok{(}\AttributeTok{x =}\NormalTok{ day,}\AttributeTok{y =}\NormalTok{ hour,}\AttributeTok{fill=}\NormalTok{Generation\_kW))}\SpecialCharTok{+}
  \FunctionTok{geom\_tile}\NormalTok{(}\AttributeTok{color=} \StringTok{"white"}\NormalTok{,}\AttributeTok{size=}\FloatTok{0.1}\NormalTok{) }\SpecialCharTok{+} 
  \FunctionTok{scale\_fill\_viridis}\NormalTok{(}\AttributeTok{name=}\StringTok{"Houerly PV Generation [kWh]"}\NormalTok{,}\AttributeTok{option =}\StringTok{"C"}\NormalTok{) }\SpecialCharTok{+} 
  \FunctionTok{facet\_grid}\NormalTok{(year }\SpecialCharTok{\textasciitilde{}}\NormalTok{ month\_label) }\SpecialCharTok{+}
  \FunctionTok{scale\_y\_continuous}\NormalTok{(}\AttributeTok{trans =} \StringTok{"reverse"}\NormalTok{, }\AttributeTok{breaks =} \FunctionTok{unique}\NormalTok{(df\_A\_joined}\SpecialCharTok{$}\NormalTok{hour)) }\SpecialCharTok{+}
  \FunctionTok{scale\_x\_continuous}\NormalTok{(}\AttributeTok{breaks =}\FunctionTok{c}\NormalTok{(}\DecValTok{1}\NormalTok{,}\DecValTok{10}\NormalTok{,}\DecValTok{20}\NormalTok{,}\DecValTok{31}\NormalTok{)) }\SpecialCharTok{+}
  \FunctionTok{theme\_minimal}\NormalTok{(}\AttributeTok{base\_size =} \DecValTok{10}\NormalTok{)}\SpecialCharTok{+}
  \FunctionTok{labs}\NormalTok{(}\AttributeTok{title=} \StringTok{"Seasonal and Timely Distribution of PV{-}Plant Production"}\NormalTok{)}\SpecialCharTok{+} 
  \FunctionTok{theme}\NormalTok{(}\AttributeTok{legend.position =} \StringTok{"bottom"}\NormalTok{) }\SpecialCharTok{+}
  \FunctionTok{theme}\NormalTok{(}\AttributeTok{plot.title=}\FunctionTok{element\_text}\NormalTok{(}\AttributeTok{size =} \DecValTok{14}\NormalTok{)) }\SpecialCharTok{+} \FunctionTok{theme}\NormalTok{(}\AttributeTok{axis.text.y=}\FunctionTok{element\_text}\NormalTok{(}\AttributeTok{size=}\DecValTok{6}\NormalTok{)) }\SpecialCharTok{+}
  \FunctionTok{theme}\NormalTok{(}\AttributeTok{strip.background =} \FunctionTok{element\_rect}\NormalTok{(}\AttributeTok{colour=}\StringTok{"white"}\NormalTok{)) }\SpecialCharTok{+}
  \FunctionTok{theme}\NormalTok{(}\AttributeTok{plot.title=}\FunctionTok{element\_text}\NormalTok{(}\AttributeTok{hjust=}\DecValTok{0}\NormalTok{)) }\SpecialCharTok{+} \FunctionTok{theme}\NormalTok{(}\AttributeTok{axis.ticks=}\FunctionTok{element\_blank}\NormalTok{()) }\SpecialCharTok{+}
  \FunctionTok{theme}\NormalTok{(}\AttributeTok{axis.text=}\FunctionTok{element\_text}\NormalTok{(}\AttributeTok{size=}\DecValTok{7}\NormalTok{)) }\SpecialCharTok{+} \FunctionTok{theme}\NormalTok{(}\AttributeTok{legend.title=}\FunctionTok{element\_text}\NormalTok{(}\AttributeTok{size=}\DecValTok{8}\NormalTok{)) }\SpecialCharTok{+}
  \FunctionTok{theme}\NormalTok{(}\AttributeTok{legend.text=}\FunctionTok{element\_text}\NormalTok{(}\AttributeTok{size=}\DecValTok{6}\NormalTok{)) }\SpecialCharTok{+}\FunctionTok{theme}\NormalTok{(}\AttributeTok{plot.title =} \FunctionTok{element\_text}\NormalTok{(}\AttributeTok{hjust =} \FloatTok{0.5}\NormalTok{)) }\SpecialCharTok{+}
  \FunctionTok{removeGrid}\NormalTok{()}
\NormalTok{p6}
\end{Highlighting}
\end{Shaded}

\includegraphics{Groupwork_Andréoli_Bigler_files/figure-latex/unnamed-chunk-16-1.pdf}

\begin{itemize}
\tightlist
\item
  the plot above shows the seasonal and timely distribution of the PV
  production of the plant A.
\end{itemize}

\newpage

\hypertarget{battery-loading-algorithm}{%
\section{Battery Loading Algorithm}\label{battery-loading-algorithm}}

The main drawback of photovoltaics is that most of the energy production
falls into the summer months, which leads to a surplus of energy during
this period. This surplus can be stored with a battery, for example. We
would like to create a simple battery charging algorithm to simulate a
battery in the solar system and evaluate how the self-consumption-rate
can be increased by adding a storage.

\hypertarget{self-consumption-rate-without-storage-system}{%
\subsection{Self-Consumption-Rate Without Storage
System}\label{self-consumption-rate-without-storage-system}}

Lets first calculate the self-consumption-rate of the system without a
battery. We define a function as follows to determine when the produced
energy is directly consumed and not feed into the grid:

\begin{Shaded}
\begin{Highlighting}[]
\NormalTok{self\_consumption\_rate }\OtherTok{\textless{}{-}} \ControlFlowTok{function}\NormalTok{(generation,consumption,i)\{}
\NormalTok{  self\_con }\OtherTok{\textless{}{-}} \FunctionTok{as.data.frame}\NormalTok{(}\FunctionTok{matrix}\NormalTok{(}\DecValTok{1}\SpecialCharTok{:}\DecValTok{8759}\NormalTok{,}\AttributeTok{ncol=}\DecValTok{1}\NormalTok{))}
    \ControlFlowTok{if}\NormalTok{ ((generation }\SpecialCharTok{{-}}\NormalTok{ consumption) }\SpecialCharTok{\textgreater{}=} \DecValTok{0}\NormalTok{) \{}
\NormalTok{      self\_con}\SpecialCharTok{$}\NormalTok{V1[i] }\OtherTok{=}\NormalTok{ consumption}
\NormalTok{    \}}\ControlFlowTok{else}\NormalTok{ \{}
\NormalTok{      self\_con}\SpecialCharTok{$}\NormalTok{V1[i] }\OtherTok{=} \DecValTok{0}
\NormalTok{    \}}
\NormalTok{\}  }
\end{Highlighting}
\end{Shaded}

We create then a new data frame self\_consumption and apply the function
with a for loop over the original data frame df\_A\_joined as follows:

\begin{Shaded}
\begin{Highlighting}[]
\NormalTok{self\_consumption }\OtherTok{\textless{}{-}} \FunctionTok{as.data.frame}\NormalTok{(}\FunctionTok{matrix}\NormalTok{(}\DecValTok{1}\SpecialCharTok{:}\DecValTok{8759}\NormalTok{,}\AttributeTok{ncol=}\DecValTok{1}\NormalTok{))}
\ControlFlowTok{for}\NormalTok{(i }\ControlFlowTok{in} \DecValTok{1}\SpecialCharTok{:}\FunctionTok{nrow}\NormalTok{(df\_A\_joined))\{}
\NormalTok{    self\_consumption}\SpecialCharTok{$}\NormalTok{V1[i] }\OtherTok{\textless{}{-}} \FunctionTok{self\_consumption\_rate}\NormalTok{(df\_A\_joined}\SpecialCharTok{$}\NormalTok{Generation\_kW[i],}
\NormalTok{                                                df\_A\_joined}\SpecialCharTok{$}\NormalTok{Overall\_Consumption\_Calc\_kW[i],}
\NormalTok{                                                i)}
\NormalTok{\}}
\end{Highlighting}
\end{Shaded}

Now we can calculate the self-consumption-rate as follows:

\begin{Shaded}
\begin{Highlighting}[]
\FunctionTok{sum}\NormalTok{(self\_consumption}\SpecialCharTok{$}\NormalTok{V1)}\SpecialCharTok{/}\FunctionTok{sum}\NormalTok{(df\_A\_joined}\SpecialCharTok{$}\NormalTok{Generation\_kW)}
\end{Highlighting}
\end{Shaded}

\begin{verbatim}
## [1] 0.199887
\end{verbatim}

\begin{itemize}
\tightlist
\item
  we can see that only 20 \% of the produced energy is used directly,
  the other part is feed into the grid.
\end{itemize}

\hypertarget{create-simple-battery-loading-algorithm}{%
\subsubsection{Create Simple Battery Loading
Algorithm}\label{create-simple-battery-loading-algorithm}}

The next step is to create a simple battery loading function. The input
variables for the function are battery capacity, max. battery capacity,
min. battery capacity, the PV generation as well as the consumption and
the initial battery charging state.

\begin{Shaded}
\begin{Highlighting}[]
\NormalTok{battery\_state }\OtherTok{\textless{}{-}} \ControlFlowTok{function}\NormalTok{(battery\_capacity,max\_battery\_capacity,min\_battery\_capacity,pv\_generation, consumption,battery\_state) \{}
  
\NormalTok{  min\_battery\_load }\OtherTok{=}\NormalTok{ battery\_capacity }\SpecialCharTok{*}\NormalTok{ (min\_battery\_capacity)}\SpecialCharTok{/}\DecValTok{100}
\NormalTok{  max\_battery\_load }\OtherTok{=}\NormalTok{ battery\_capacity }\SpecialCharTok{*}\NormalTok{ (max\_battery\_capacity)}\SpecialCharTok{/}\DecValTok{100}
\NormalTok{  battery\_load }\OtherTok{=}\NormalTok{ battery\_state}
  
  \ControlFlowTok{if}\NormalTok{ ((pv\_generation }\SpecialCharTok{{-}}\NormalTok{ consumption) }\SpecialCharTok{\textgreater{}} \DecValTok{0} \SpecialCharTok{\&}\NormalTok{ battery\_load }\SpecialCharTok{\textless{}}\NormalTok{ max\_battery\_load) \{}
    \ControlFlowTok{if}\NormalTok{ (battery\_load }\SpecialCharTok{+}\NormalTok{ (pv\_generation }\SpecialCharTok{{-}}\NormalTok{ consumption) }\SpecialCharTok{\textgreater{}}\NormalTok{ max\_battery\_load) \{}
\NormalTok{      battery\_load }\OtherTok{=}\NormalTok{ max\_battery\_load  }
\NormalTok{    \} }\ControlFlowTok{else} 
\NormalTok{    \{battery\_load }\OtherTok{=}\NormalTok{ battery\_load }\SpecialCharTok{+}\NormalTok{ (pv\_generation }\SpecialCharTok{{-}}\NormalTok{ consumption)\} }
\NormalTok{  \}}
  \ControlFlowTok{else} \ControlFlowTok{if}\NormalTok{ ((pv\_generation}\SpecialCharTok{{-}}\NormalTok{consumption) }\SpecialCharTok{\textless{}} \DecValTok{0} \SpecialCharTok{\&}\NormalTok{ (battery\_load }\SpecialCharTok{\textgreater{}}\NormalTok{ min\_battery\_load)) \{}
    \ControlFlowTok{if}\NormalTok{ (battery\_load }\SpecialCharTok{+}\NormalTok{ (pv\_generation }\SpecialCharTok{{-}}\NormalTok{ consumption) }\SpecialCharTok{\textless{}}\NormalTok{ min\_battery\_load) \{}
\NormalTok{      battery\_load }\OtherTok{=}\NormalTok{ min\_battery\_load }
\NormalTok{    \} }\ControlFlowTok{else}\NormalTok{ \{}
\NormalTok{      battery\_load }\OtherTok{=}\NormalTok{ battery\_load }\SpecialCharTok{+}\NormalTok{ (pv\_generation }\SpecialCharTok{{-}}\NormalTok{ consumption) \}}
\NormalTok{  \}}
   \FunctionTok{return}\NormalTok{(battery\_load) }
\NormalTok{\}  }
\end{Highlighting}
\end{Shaded}

We apply now the battery loading algorithm. We us different battery
sizes from 45 kWH up to 450 kWh in 45 kWh steps.

\begin{Shaded}
\begin{Highlighting}[]
\NormalTok{max\_battery\_capacity }\OtherTok{\textless{}{-}} \DecValTok{95} \CommentTok{\# in \%}
\NormalTok{min\_battery\_capacity }\OtherTok{\textless{}{-}} \DecValTok{15} \CommentTok{\# in \%}

\NormalTok{battery\_states }\OtherTok{\textless{}{-}} \FunctionTok{as.data.frame}\NormalTok{(}\FunctionTok{matrix}\NormalTok{(}\DecValTok{0}\NormalTok{,}\AttributeTok{nrow=}\DecValTok{8759}\NormalTok{,}\AttributeTok{ncol=}\DecValTok{10}\NormalTok{))}
\NormalTok{x }\OtherTok{\textless{}{-}} \DecValTok{1}

\ControlFlowTok{for}\NormalTok{ (j }\ControlFlowTok{in} \FunctionTok{seq}\NormalTok{(}\DecValTok{45}\NormalTok{, }\DecValTok{450}\NormalTok{, }\AttributeTok{length.out=}\DecValTok{10}\NormalTok{)) \{}
    \ControlFlowTok{for}\NormalTok{(i }\ControlFlowTok{in} \DecValTok{2}\SpecialCharTok{:}\FunctionTok{nrow}\NormalTok{(df\_A\_joined)) \{ }
\NormalTok{      battery\_states[i,x] }\OtherTok{\textless{}{-}} \FunctionTok{battery\_state}\NormalTok{(j,max\_battery\_capacity,min\_battery\_capacity,}
\NormalTok{                                                df\_A\_joined}\SpecialCharTok{$}\NormalTok{Generation\_kW[i],}
\NormalTok{                                                df\_A\_joined}\SpecialCharTok{$}\NormalTok{Overall\_Consumption\_Calc\_kW[i],}
\NormalTok{                                                battery\_states[i}\DecValTok{{-}1}\NormalTok{,x])}
\NormalTok{    \}}
\NormalTok{  x }\OtherTok{=}\NormalTok{ x }\SpecialCharTok{+} \DecValTok{1}
\NormalTok{\}}
\end{Highlighting}
\end{Shaded}

\hypertarget{calculate-self-consumption-rate-for-differnt-battery-sizes}{%
\subsubsection{Calculate Self-Consumption-Rate for Differnt Battery
Sizes}\label{calculate-self-consumption-rate-for-differnt-battery-sizes}}

\begin{Shaded}
\begin{Highlighting}[]
\NormalTok{self\_consumption\_battery }\OtherTok{\textless{}{-}} \FunctionTok{as.data.frame}\NormalTok{(}\FunctionTok{matrix}\NormalTok{(}\DecValTok{0}\NormalTok{,}\AttributeTok{nrow=}\DecValTok{8759}\NormalTok{,}\AttributeTok{ncol=}\DecValTok{10}\NormalTok{))}
\ControlFlowTok{for}\NormalTok{(j }\ControlFlowTok{in} \FunctionTok{seq}\NormalTok{(}\DecValTok{1}\NormalTok{,}\DecValTok{10}\NormalTok{,}\AttributeTok{length.out =} \DecValTok{10}\NormalTok{))\{}
  \ControlFlowTok{for}\NormalTok{(i }\ControlFlowTok{in} \DecValTok{1}\SpecialCharTok{:}\DecValTok{8758}\NormalTok{) \{}
    \ControlFlowTok{if}\NormalTok{ ((battery\_states[i}\SpecialCharTok{+}\DecValTok{1}\NormalTok{,j] }\SpecialCharTok{{-}}\NormalTok{ battery\_states[i,j]) }\SpecialCharTok{\textless{}} \DecValTok{0}\NormalTok{) \{}
\NormalTok{      self\_consumption\_battery[i}\SpecialCharTok{+}\DecValTok{1}\NormalTok{,j] }\OtherTok{\textless{}{-}}\NormalTok{ self\_consumption[i}\SpecialCharTok{+}\DecValTok{1}\NormalTok{,}\DecValTok{1}\NormalTok{] }\SpecialCharTok{+} \FunctionTok{abs}\NormalTok{(battery\_states[i}\SpecialCharTok{+}\DecValTok{1}\NormalTok{,j] }\SpecialCharTok{{-}}\NormalTok{ battery\_states[i,j])}
\NormalTok{    \} }\ControlFlowTok{else}\NormalTok{\{self\_consumption\_battery[i}\SpecialCharTok{+}\DecValTok{1}\NormalTok{,j] }\OtherTok{\textless{}{-}}\NormalTok{ self\_consumption[i}\SpecialCharTok{+}\DecValTok{1}\NormalTok{,}\DecValTok{1}\NormalTok{]\}}
\NormalTok{  \}}
\NormalTok{\}}
\end{Highlighting}
\end{Shaded}

\begin{itemize}
\tightlist
\item
  The next step is to create a dataframe containing the different
  self-consumption-ratios and the corresponding battery capacity.
\end{itemize}

\newpage

\begin{Shaded}
\begin{Highlighting}[]
\NormalTok{df\_self\_con\_ratio }\OtherTok{\textless{}{-}} \FunctionTok{as.data.frame}\NormalTok{(}\FunctionTok{colSums}\NormalTok{(self\_consumption\_battery)}\SpecialCharTok{/}\FunctionTok{sum}\NormalTok{(df\_A\_joined}\SpecialCharTok{$}\NormalTok{Generation\_kW))}

\NormalTok{df\_self\_con\_ratio}\SpecialCharTok{$}\NormalTok{battery\_capacity }\OtherTok{\textless{}{-}} \FunctionTok{seq}\NormalTok{(}\DecValTok{45}\NormalTok{, }\DecValTok{450}\NormalTok{, }\AttributeTok{length.out=}\DecValTok{10}\NormalTok{)}
\FunctionTok{colnames}\NormalTok{(df\_self\_con\_ratio) }\OtherTok{\textless{}{-}} \FunctionTok{c}\NormalTok{(}\StringTok{"self\_consumption\_ratio\_kWh"}\NormalTok{,}\StringTok{"battery\_capacity\_kWh"}\NormalTok{)}

\NormalTok{new\_row }\OtherTok{\textless{}{-}} \FunctionTok{c}\NormalTok{(}\FunctionTok{sum}\NormalTok{(self\_consumption}\SpecialCharTok{$}\NormalTok{V1)}\SpecialCharTok{/}\FunctionTok{sum}\NormalTok{(df\_A\_joined}\SpecialCharTok{$}\NormalTok{Generation\_kW),}\DecValTok{0}\NormalTok{)}
\NormalTok{df\_self\_con\_ratio  }\OtherTok{\textless{}{-}} \FunctionTok{rbind}\NormalTok{(new\_row,df\_self\_con\_ratio)}
\end{Highlighting}
\end{Shaded}

\begin{verbatim}
## <ScaleContinuousPosition>
##  Range:  
##  Limits:    0 --  500
\end{verbatim}

\begin{verbatim}
## `geom_smooth()` using method = 'loess' and formula 'y ~ x'
\end{verbatim}

\includegraphics{Groupwork_Andréoli_Bigler_files/figure-latex/unnamed-chunk-24-1.pdf}

\begin{itemize}
\tightlist
\item
  as we can see from the plot above, the self-consumption-rate starts to
  flatten out at around a capacity of 220 kWh. This can be explained by
  the fact, that after a certain point, an increase in battery capacity
  no longer causes a noticeable increase in the self-consumption rate,
  since only individual peaks of the PV production are additionally
  stored.
\end{itemize}

In the next chapters we try to fit a timeseries model and a GAM model to
predict the PV production.

\newpage

\hypertarget{time-series-analysis---with-auto.arima}{%
\subsection{Time Series Analysis - With
auto.arima}\label{time-series-analysis---with-auto.arima}}

The explored model for prediction is a time series model, called SARIMA.
To be able to compute those types of models, the data set has first to
be converted into a time series object. As the data has an hourly
resolution, the time interval of the TS object is set to 24, which
correspond, in this case, to the seasonality of one day. The model is
then trained with the first 250 days of the year.

\begin{Shaded}
\begin{Highlighting}[]
\FunctionTok{library}\NormalTok{(forecast)}
\end{Highlighting}
\end{Shaded}

\begin{verbatim}
## Warning: package 'forecast' was built under R version 4.0.5
\end{verbatim}

\begin{verbatim}
## Registered S3 method overwritten by 'quantmod':
##   method            from
##   as.zoo.data.frame zoo
\end{verbatim}

\begin{verbatim}
## 
## Attaching package: 'forecast'
\end{verbatim}

\begin{verbatim}
## The following object is masked from 'package:nlme':
## 
##     getResponse
\end{verbatim}

\begin{Shaded}
\begin{Highlighting}[]
\NormalTok{ts\_1 }\OtherTok{\textless{}{-}} \FunctionTok{ts}\NormalTok{((df\_plantA\_resample}\SpecialCharTok{$}\NormalTok{Generation\_kW), }\AttributeTok{deltat =} \DecValTok{1}\SpecialCharTok{/}\DecValTok{24}\NormalTok{)}
\NormalTok{train }\OtherTok{\textless{}{-}} \FunctionTok{window}\NormalTok{(ts\_1, }\AttributeTok{start =} \DecValTok{1}\NormalTok{, }\AttributeTok{end =} \DecValTok{250}\NormalTok{)}
\NormalTok{fit }\OtherTok{\textless{}{-}} \FunctionTok{auto.arima}\NormalTok{(train)}
\end{Highlighting}
\end{Shaded}

\hypertarget{model-prediction}{%
\subsubsection{Model Prediction}\label{model-prediction}}

The trained model is then used for a prediction. The predicted days are
from day 250 to 365. The result can be seen in the following plot. Red
are the predicted data, black are the real data.

\includegraphics{Groupwork_Andréoli_Bigler_files/figure-latex/unnamed-chunk-26-1.pdf}

As one can see above, prediction range of the plot is not very helpful.
So, another plot is computed, but this time with a zoom set on the
prediction area. As it is to see, the prediction has some variation for
the first few days. Then, it converges to value which is a little bit
higher than the mean of the data.

\begin{Shaded}
\begin{Highlighting}[]
\NormalTok{fc }\OtherTok{\textless{}{-}} \FunctionTok{predict}\NormalTok{(fit, }\AttributeTok{n.ahead =} \DecValTok{115}\SpecialCharTok{*}\DecValTok{24}\NormalTok{)}
\FunctionTok{plot}\NormalTok{(ts\_1, }\AttributeTok{lty=}\DecValTok{3}\NormalTok{, }\AttributeTok{cex =} \FloatTok{0.1}\NormalTok{, }\AttributeTok{xlim=}\FunctionTok{c}\NormalTok{(}\DecValTok{240}\NormalTok{, }\DecValTok{270}\NormalTok{), }\AttributeTok{ylim=}\FunctionTok{c}\NormalTok{(}\DecValTok{25}\NormalTok{, }\DecValTok{125}\NormalTok{))}
\FunctionTok{lines}\NormalTok{(train, }\AttributeTok{lwd=}\DecValTok{1}\NormalTok{)}
\FunctionTok{lines}\NormalTok{(fc}\SpecialCharTok{$}\NormalTok{pred, }\AttributeTok{lwd=}\DecValTok{2}\NormalTok{, }\AttributeTok{col=}\StringTok{"red"}\NormalTok{, }\AttributeTok{cex =} \FloatTok{0.1}\NormalTok{)}
\end{Highlighting}
\end{Shaded}

\includegraphics{Groupwork_Andréoli_Bigler_files/figure-latex/unnamed-chunk-27-1.pdf}

\hypertarget{time-series-analysis---with-manipulated-data}{%
\subsection{Time Series Analysis - with Manipulated
Data}\label{time-series-analysis---with-manipulated-data}}

Because the previous results were not satisfying at all, a few changes
to the data set are undertaken. First, the resolution is set to daily.
Secondly, some artificial years are generated and added to the time
series. This, in order to be able to capture the seasonal effect over
the entire year. This entire part is done with an experimental intention
and from curiosity. With the resolution set to daily, the data set
shrink to 365 entries.

Plot of new time series with a resolution of one day and a GAM smoother.

\begin{Shaded}
\begin{Highlighting}[]
\FunctionTok{ggplot}\NormalTok{(}\AttributeTok{data =}\NormalTok{ df\_plantA\_resample\_2,}
  \AttributeTok{mapping =} \FunctionTok{aes}\NormalTok{(}\AttributeTok{y =}\NormalTok{ Generation\_kW, }\AttributeTok{x =}\NormalTok{ datetime)) }\SpecialCharTok{+}
  \FunctionTok{geom\_point}\NormalTok{(}\AttributeTok{size =} \DecValTok{1}\NormalTok{, }\AttributeTok{color =} \StringTok{"grey69"}\NormalTok{) }\SpecialCharTok{+}
  \FunctionTok{geom\_smooth}\NormalTok{(}\AttributeTok{method =} \StringTok{"gam"}\NormalTok{, }\AttributeTok{color =} \StringTok{"cornflowerblue"}\NormalTok{)}
\end{Highlighting}
\end{Shaded}

\begin{verbatim}
## `geom_smooth()` using formula 'y ~ s(x, bs = "cs")'
\end{verbatim}

\includegraphics{Groupwork_Andréoli_Bigler_files/figure-latex/unnamed-chunk-29-1.pdf}

First, some artificial years were created by simply adding noise with
the jitter() function. Because of the occurrence of minus values in this
procedure, minus values have to be corrected to zero in a second
process. With the new data set, a TS decomposition is plotted via the
function stl(). Then, a new model is trained and a forecast is
predicted. Again, the prediction performance is rather poor.

\begin{Shaded}
\begin{Highlighting}[]
\NormalTok{ts\_artificial }\OtherTok{\textless{}{-}} \FunctionTok{ts}\NormalTok{(}\FunctionTok{c}\NormalTok{(ts\_2, ts\_artif\_1, ts\_artif\_2), }\AttributeTok{start =} \FunctionTok{c}\NormalTok{(}\DecValTok{2019}\NormalTok{), }\AttributeTok{deltat =} \DecValTok{1}\SpecialCharTok{/}\DecValTok{365}\NormalTok{)}
\NormalTok{decomp}\OtherTok{\textless{}{-}}\FunctionTok{stl}\NormalTok{(ts\_artificial, }\AttributeTok{s.window =} \DecValTok{365}\NormalTok{)}
\FunctionTok{str}\NormalTok{(ts\_artificial)}
\end{Highlighting}
\end{Shaded}

\begin{verbatim}
##  Time-Series [1:1095] from 2019 to 2022: 71.26 226.41 193.9 242.04 1.25 ...
\end{verbatim}

\begin{Shaded}
\begin{Highlighting}[]
\FunctionTok{plot}\NormalTok{(decomp)}
\end{Highlighting}
\end{Shaded}

\includegraphics{Groupwork_Andréoli_Bigler_files/figure-latex/unnamed-chunk-31-1.pdf}

Train model with the first two years of the time series. To train the
model, the auto.arima function is made use of.

\includegraphics{Groupwork_Andréoli_Bigler_files/figure-latex/unnamed-chunk-32-1.pdf}

\hypertarget{time-series-analysis---with-manipulated-data---new-approach}{%
\subsection{Time Series Analysis - with Manipulated Data - new
approach}\label{time-series-analysis---with-manipulated-data---new-approach}}

In the previous part, the creation of the artificial years brought up
some concerns regarding the quality of those artificial data. In this
part, the focus is set to discover a method to generate artificial data
similar to the real data in terms of variance, mean and distribution of
the data. The method is as follow: : 1. Use of GAM smoother as base for
an artificial year 2. Add normally distributed, left screwed noise.

For that, a new data frame with indexed data is created. This is done to
avoid the handling of date type.

\begin{Shaded}
\begin{Highlighting}[]
\NormalTok{df\_artif }\OtherTok{\textless{}{-}} \FunctionTok{data.frame}\NormalTok{(}\AttributeTok{time =}\NormalTok{ (}\DecValTok{1}\SpecialCharTok{:}\DecValTok{365}\NormalTok{), }\AttributeTok{generation\_kw =}\NormalTok{ df\_plantA\_resample\_2}\SpecialCharTok{$}\NormalTok{Generation\_kW)}
\end{Highlighting}
\end{Shaded}

Now, a GAM is computed from the new data frame.

\begin{Shaded}
\begin{Highlighting}[]
\NormalTok{gam\_model }\OtherTok{\textless{}{-}} \FunctionTok{gam}\NormalTok{(generation\_kw }\SpecialCharTok{\textasciitilde{}} \FunctionTok{s}\NormalTok{(time), }\AttributeTok{data =}\NormalTok{ df\_artif)}
\end{Highlighting}
\end{Shaded}

In order to generate a data set out of the GAM, 365 prediction are
computed within the entire year. The result is plotted to visually check
the result. As shown below, the data frame plot is, at least
qualitatively, very similar to the ggplot GAM smoother line. This result
is taken as a base for the artificial years.

\begin{Shaded}
\begin{Highlighting}[]
\NormalTok{df\_time }\OtherTok{\textless{}{-}} \FunctionTok{data.frame}\NormalTok{(}\AttributeTok{time =} \FunctionTok{c}\NormalTok{(}\DecValTok{1}\SpecialCharTok{:}\DecValTok{365}\NormalTok{))}
\NormalTok{gam\_prediction }\OtherTok{\textless{}{-}} \FunctionTok{predict}\NormalTok{(gam\_model, }\AttributeTok{newdata =}\NormalTok{ df\_time)}
\FunctionTok{plot}\NormalTok{(gam\_prediction, }\AttributeTok{cex =} \FloatTok{0.1}\NormalTok{, )}
\end{Highlighting}
\end{Shaded}

\includegraphics{Groupwork_Andréoli_Bigler_files/figure-latex/unnamed-chunk-35-1.pdf}

After the creation of a ``base year'', noise is added. In order to keep
the heteroskedastic behavior of the real data, the noise is added as a
multiplication by a random coefficient. This coefficient is first
created as a left screwed, normally distributed random number. Then, it
is normalized from 0 to 1. Last, those noise coefficient between 0 and 1
are again multiplied with an random coefficient in a predefined range
(0.3 to 1.6), in order to level the mean of the data set. A quick look
at the plot show a satisfying result.

Add noise to smoothed base year:

\begin{Shaded}
\begin{Highlighting}[]
\FunctionTok{set.seed}\NormalTok{(}\DecValTok{4}\NormalTok{)}
\NormalTok{random\_coef }\OtherTok{\textless{}{-}} \FunctionTok{rsnorm}\NormalTok{(}\DecValTok{365}\NormalTok{, }\AttributeTok{mean =} \DecValTok{1}\NormalTok{, }\AttributeTok{sd =} \DecValTok{1}\NormalTok{, }\AttributeTok{xi =} \FloatTok{0.1}\NormalTok{)}
\NormalTok{ran\_coef\_norm }\OtherTok{\textless{}{-}} \FunctionTok{normalize}\NormalTok{(random\_coef, }\AttributeTok{method =} \StringTok{"range"}\NormalTok{, }\AttributeTok{range =} \FunctionTok{c}\NormalTok{(}\DecValTok{0}\NormalTok{, }\DecValTok{1}\NormalTok{))}
\NormalTok{ran\_coef\_norm\_2 }\OtherTok{\textless{}{-}}\NormalTok{ ran\_coef\_norm }\SpecialCharTok{*} \FunctionTok{runif}\NormalTok{(ran\_coef\_norm, }\AttributeTok{min =} \FloatTok{0.3}\NormalTok{, }\AttributeTok{max =} \FloatTok{1.6}\NormalTok{)}
\NormalTok{gam\_yr\_with\_noise }\OtherTok{\textless{}{-}}\NormalTok{ gam\_prediction }\SpecialCharTok{*}\NormalTok{ ran\_coef\_norm\_2}
\end{Highlighting}
\end{Shaded}

\begin{verbatim}
## [1] "Real Data"
\end{verbatim}

\begin{verbatim}
## [1] "Mean:      684.24677260274"
\end{verbatim}

\begin{verbatim}
## [1] "Variance:  232351.304469434"
\end{verbatim}

\begin{verbatim}
## [1] "GAM prediction"
\end{verbatim}

\begin{verbatim}
## [1] "Mean:      684.246772602739"
\end{verbatim}

\begin{verbatim}
## [1] "Variance:  159111.627971204"
\end{verbatim}

\begin{verbatim}
## [1] "Artificial Data"
\end{verbatim}

\begin{verbatim}
## [1] "Mean:      504.585754013336"
\end{verbatim}

\begin{verbatim}
## [1] "Variance:  161066.585646483"
\end{verbatim}

\includegraphics{Groupwork_Andréoli_Bigler_files/figure-latex/unnamed-chunk-37-1.pdf}

While the artificial year visually looks satisfying, the mean generally
is too low. It is not easy to change the ``noise'' parameters in such a
way, that the variance and the mean of the artificial year gets near the
similar value of the real data. Therefore, a function is create which
aims to find the best parameter setting:

First, a list of parameter sets is created. This list is then used in a
function that loops through the list and creates, for each set, a noised
year. The noised year is then compared with the real year by its mean
and variance. A threshold for mean and variance is set and if they are
cumulatively fulfilled, the function saves the year data in a list. At
the end of the loop, the list of years is returned by the function.

\hypertarget{function-to-get-artificial-year}{%
\subsubsection{Function To Get Artificial
Year}\label{function-to-get-artificial-year}}

Create list of parameter sets:

\begin{Shaded}
\begin{Highlighting}[]
\FunctionTok{library}\NormalTok{(BBmisc)}
\NormalTok{xi }\OtherTok{\textless{}{-}} \FunctionTok{c}\NormalTok{(}\DecValTok{25}\SpecialCharTok{:}\DecValTok{1}\NormalTok{)}\SpecialCharTok{/}\DecValTok{30}
\NormalTok{min }\OtherTok{\textless{}{-}} \FunctionTok{c}\NormalTok{(}\DecValTok{1}\SpecialCharTok{:}\DecValTok{50}\NormalTok{)}\SpecialCharTok{/}\DecValTok{100}
\NormalTok{max }\OtherTok{\textless{}{-}} \FunctionTok{c}\NormalTok{(}\DecValTok{60}\SpecialCharTok{:}\DecValTok{90}\NormalTok{)}\SpecialCharTok{/}\DecValTok{40}
\NormalTok{u }\OtherTok{\textless{}{-}} \FunctionTok{list}\NormalTok{()}
  \ControlFlowTok{for}\NormalTok{ (i }\ControlFlowTok{in}\NormalTok{ xi) \{}
    \ControlFlowTok{for}\NormalTok{ (n }\ControlFlowTok{in}\NormalTok{ min) \{}
      \ControlFlowTok{for}\NormalTok{ (q }\ControlFlowTok{in}\NormalTok{ max) \{}
\NormalTok{        o }\OtherTok{\textless{}{-}} \FunctionTok{c}\NormalTok{(i, n, q)}
\NormalTok{        u }\OtherTok{\textless{}{-}} \FunctionTok{rbind}\NormalTok{(u,o)}
\NormalTok{      \}}
\NormalTok{    \}}
\NormalTok{  \}}
\end{Highlighting}
\end{Shaded}

Following is the function that loops through parameter list:

\begin{Shaded}
\begin{Highlighting}[]
\NormalTok{get\_artificial\_years }\OtherTok{\textless{}{-}} \ControlFlowTok{function}\NormalTok{(mean\_real\_yr, variance\_real\_yr, gam\_prediction) \{}
  
\NormalTok{  div\_mean }\OtherTok{\textless{}{-}} \DecValTok{0}
\NormalTok{  div\_var }\OtherTok{\textless{}{-}} \DecValTok{0}
\NormalTok{  i }\OtherTok{=} \DecValTok{1}
\NormalTok{  treshold\_1 }\OtherTok{\textless{}{-}} \ConstantTok{TRUE}
\NormalTok{  treshold\_2 }\OtherTok{\textless{}{-}} \ConstantTok{TRUE}
\NormalTok{  df\_sets }\OtherTok{\textless{}{-}} \FunctionTok{list}\NormalTok{()}
\NormalTok{  m }\OtherTok{\textless{}{-}}\NormalTok{ (}\FunctionTok{length}\NormalTok{(u)}\SpecialCharTok{/}\DecValTok{3} \SpecialCharTok{{-}} \DecValTok{1}\NormalTok{)}
  
  \ControlFlowTok{for}\NormalTok{ (n }\ControlFlowTok{in}\NormalTok{ u) \{}
    \ControlFlowTok{if}\NormalTok{ (i }\SpecialCharTok{==}\NormalTok{ m) \{}
      \ControlFlowTok{break}
\NormalTok{    \}}
\NormalTok{    ran\_coef }\OtherTok{\textless{}{-}} \FunctionTok{rsnorm}\NormalTok{(}\DecValTok{365}\NormalTok{, }\AttributeTok{mean =} \DecValTok{1}\NormalTok{, }\AttributeTok{sd =} \DecValTok{1}\NormalTok{, }\AttributeTok{xi =} \FunctionTok{as.numeric}\NormalTok{(u[i,][}\DecValTok{1}\NormalTok{]))}
\NormalTok{    ran\_coef\_norm }\OtherTok{\textless{}{-}} \FunctionTok{normalize}\NormalTok{(ran\_coef, }\AttributeTok{method =} \StringTok{"range"}\NormalTok{, }\AttributeTok{range =} \FunctionTok{c}\NormalTok{(}\DecValTok{0}\NormalTok{, }\DecValTok{1}\NormalTok{))}
\NormalTok{    ran\_coef\_norm\_2 }\OtherTok{\textless{}{-}}\NormalTok{ ran\_coef\_norm }\SpecialCharTok{*} \FunctionTok{runif}\NormalTok{(ran\_coef\_norm, }\AttributeTok{min =} \FunctionTok{as.numeric}\NormalTok{(u[i,][}\DecValTok{2}\NormalTok{]), }\AttributeTok{max =} \FunctionTok{as.numeric}\NormalTok{(u[i,][}\DecValTok{3}\NormalTok{]))}

\NormalTok{    yr\_with\_noise }\OtherTok{\textless{}{-}}\NormalTok{ gam\_prediction }\SpecialCharTok{*}\NormalTok{ ran\_coef\_norm\_2}
\NormalTok{    div\_mean }\OtherTok{\textless{}{-}}\NormalTok{ (mean\_real\_yr }\SpecialCharTok{/} \FunctionTok{mean}\NormalTok{(yr\_with\_noise))}
\NormalTok{    div\_var }\OtherTok{\textless{}{-}}\NormalTok{ (variance\_real\_yr }\SpecialCharTok{/} \FunctionTok{var}\NormalTok{(yr\_with\_noise))}
       
\NormalTok{    treshold\_1 }\OtherTok{\textless{}{-}}\NormalTok{ (div\_mean }\SpecialCharTok{\textless{}} \FloatTok{1.025}  \SpecialCharTok{\&\&}\NormalTok{ div\_mean }\SpecialCharTok{\textgreater{}} \FloatTok{0.95}\NormalTok{)}
\NormalTok{    treshold\_2 }\OtherTok{\textless{}{-}}\NormalTok{ (div\_var }\SpecialCharTok{\textless{}} \FloatTok{1.05} \SpecialCharTok{\&\&}\NormalTok{ div\_var }\SpecialCharTok{\textgreater{}} \FloatTok{0.95}\NormalTok{)}
    
    \ControlFlowTok{if}\NormalTok{ (treshold\_1 }\SpecialCharTok{\&\&}\NormalTok{ treshold\_2) \{}
\NormalTok{      df\_sets }\OtherTok{\textless{}{-}} \FunctionTok{rbind}\NormalTok{(df\_sets, yr\_with\_noise)}
      \FunctionTok{print}\NormalTok{(i)}
      \FunctionTok{print}\NormalTok{(div\_mean)}
      \FunctionTok{print}\NormalTok{(div\_var)}
\NormalTok{    \}}
    \ControlFlowTok{if}\NormalTok{ (i }\SpecialCharTok{\%in\%}\NormalTok{ (}\FunctionTok{c}\NormalTok{(}\DecValTok{1}\SpecialCharTok{:}\DecValTok{10000}\NormalTok{)}\SpecialCharTok{*}\DecValTok{5000}\NormalTok{))\{}
      \FunctionTok{paste}\NormalTok{(}\StringTok{"Year computed: "}\NormalTok{, i)}
\NormalTok{    \}}
    
\NormalTok{    i }\OtherTok{\textless{}{-}}\NormalTok{ i }\SpecialCharTok{+} \DecValTok{1}
\NormalTok{  \}}
\NormalTok{  df\_sets}
\NormalTok{\}}
\end{Highlighting}
\end{Shaded}

Call the function to compute about 40'000 simulated years and and save
all ones that meet the threshold requirements. With set.seed() to 41,
four years fulfill the thresholds.

\begin{verbatim}
## [1] 15498
## [1] 0.9998943
## [1] 0.9574415
## [1] 32421
## [1] 1.018895
## [1] 0.9638508
## [1] 35515
## [1] 1.022846
## [1] 0.9602693
## [1] 35637
## [1] 1.019101
## [1] 0.9542123
\end{verbatim}

Plot and compare the results with the real data. As it is to see, the
results appear to be satisfying, excepting the few outliers, which are
values over \textasciitilde1800.

\begin{verbatim}
## [1] "Real Data"
\end{verbatim}

\begin{verbatim}
## [1] "Mean:      684.24677260274"
\end{verbatim}

\begin{verbatim}
## [1] "Variance:  232351.304469434"
\end{verbatim}

\begin{verbatim}
## [1] "Artificial Data"
\end{verbatim}

\begin{verbatim}
## [1] "Mean:      671.5575893729"
\end{verbatim}

\begin{verbatim}
## [1] "Variance:  241065.625833288"
\end{verbatim}

As last manipulation, the few outliers over a value of 1800 are pulled
down. This is done for all the returned artificial years.

\begin{Shaded}
\begin{Highlighting}[]
\NormalTok{noised\_years[}\DecValTok{1}\NormalTok{,][noised\_years[}\DecValTok{1}\NormalTok{,] }\SpecialCharTok{\textgreater{}} \DecValTok{1800}\NormalTok{] }\OtherTok{\textless{}{-}}\NormalTok{ (}\FunctionTok{as.numeric}\NormalTok{(noised\_years[}\DecValTok{1}\NormalTok{,]) }\SpecialCharTok{*} \FloatTok{0.75}\NormalTok{)}
\end{Highlighting}
\end{Shaded}

\begin{verbatim}
## Warning in noised_years[1, ][noised_years[1, ] > 1800] <-
## (as.numeric(noised_years[1, : Anzahl der zu ersetzenden Elemente ist kein
## Vielfaches der Ersetzungslänge
\end{verbatim}

\begin{Shaded}
\begin{Highlighting}[]
\NormalTok{df\_artif\_2 }\OtherTok{\textless{}{-}} \FunctionTok{t}\NormalTok{(noised\_years[}\DecValTok{1}\NormalTok{,])}
\NormalTok{array\_artif\_2 }\OtherTok{\textless{}{-}} \FunctionTok{array}\NormalTok{(}\FunctionTok{as.numeric}\NormalTok{(}\FunctionTok{unlist}\NormalTok{(df\_artif\_2)))}
\FunctionTok{par}\NormalTok{(}\AttributeTok{mfrow=}\FunctionTok{c}\NormalTok{(}\DecValTok{2}\NormalTok{,}\DecValTok{2}\NormalTok{))}
\FunctionTok{plot}\NormalTok{(array\_artif\_2, }\AttributeTok{cex =}\NormalTok{ .}\DecValTok{2}\NormalTok{, }\AttributeTok{main=}\StringTok{"Plot Artificial Data"}\NormalTok{)}
\FunctionTok{hist}\NormalTok{(array\_artif\_2, }\AttributeTok{main =} \StringTok{"Histogram Artificial Data"}\NormalTok{)}
\end{Highlighting}
\end{Shaded}

\includegraphics{Groupwork_Andréoli_Bigler_files/figure-latex/unnamed-chunk-42-1.pdf}

\begin{verbatim}
## Warning in noised_years[2, ][noised_years[2, ] > 1800] <-
## (as.numeric(noised_years[2, : Anzahl der zu ersetzenden Elemente ist kein
## Vielfaches der Ersetzungslänge
\end{verbatim}

\begin{verbatim}
## Warning in noised_years[3, ][noised_years[3, ] > 1800] <-
## (as.numeric(noised_years[3, : Anzahl der zu ersetzenden Elemente ist kein
## Vielfaches der Ersetzungslänge
\end{verbatim}

\hypertarget{plot-final-results-of-artificial-data}{%
\subsubsection{Plot final Results Of Artificial
Data}\label{plot-final-results-of-artificial-data}}

Create TS with artificial years. Then investigate with the stl()
function. Each year differs now much more than the others. This result
is considered as successful. This experimental excursion delivers an
interesting result which could be now further used to explore predictive
modelling. As for now, this will be the end of this part.

\begin{Shaded}
\begin{Highlighting}[]
\NormalTok{ts\_artificial\_2 }\OtherTok{\textless{}{-}} \FunctionTok{ts}\NormalTok{(}\FunctionTok{c}\NormalTok{(noised\_years[}\DecValTok{1}\NormalTok{,], noised\_years[}\DecValTok{2}\NormalTok{,], noised\_years[}\DecValTok{3}\NormalTok{,], noised\_years[}\DecValTok{4}\NormalTok{,]), }\AttributeTok{start =} \FunctionTok{c}\NormalTok{(}\DecValTok{2019}\NormalTok{), }\AttributeTok{deltat =} \DecValTok{1}\SpecialCharTok{/}\DecValTok{365}\NormalTok{)}
\NormalTok{decomp\_2 }\OtherTok{\textless{}{-}} \FunctionTok{stl}\NormalTok{(ts\_artificial\_2, }\AttributeTok{s.window =} \DecValTok{1}\SpecialCharTok{/}\DecValTok{24}\NormalTok{, }\AttributeTok{t.window =} \DecValTok{365}\NormalTok{)}
\FunctionTok{plot}\NormalTok{(decomp\_2)}
\end{Highlighting}
\end{Shaded}

\includegraphics{Groupwork_Andréoli_Bigler_files/figure-latex/unnamed-chunk-44-1.pdf}

\end{document}
