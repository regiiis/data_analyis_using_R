% Options for packages loaded elsewhere
\PassOptionsToPackage{unicode}{hyperref}
\PassOptionsToPackage{hyphens}{url}
%
\documentclass[
]{article}
\title{regis}
\author{regis andreoli}
\date{1/31/2022}

\usepackage{amsmath,amssymb}
\usepackage{lmodern}
\usepackage{iftex}
\ifPDFTeX
  \usepackage[T1]{fontenc}
  \usepackage[utf8]{inputenc}
  \usepackage{textcomp} % provide euro and other symbols
\else % if luatex or xetex
  \usepackage{unicode-math}
  \defaultfontfeatures{Scale=MatchLowercase}
  \defaultfontfeatures[\rmfamily]{Ligatures=TeX,Scale=1}
\fi
% Use upquote if available, for straight quotes in verbatim environments
\IfFileExists{upquote.sty}{\usepackage{upquote}}{}
\IfFileExists{microtype.sty}{% use microtype if available
  \usepackage[]{microtype}
  \UseMicrotypeSet[protrusion]{basicmath} % disable protrusion for tt fonts
}{}
\makeatletter
\@ifundefined{KOMAClassName}{% if non-KOMA class
  \IfFileExists{parskip.sty}{%
    \usepackage{parskip}
  }{% else
    \setlength{\parindent}{0pt}
    \setlength{\parskip}{6pt plus 2pt minus 1pt}}
}{% if KOMA class
  \KOMAoptions{parskip=half}}
\makeatother
\usepackage{xcolor}
\IfFileExists{xurl.sty}{\usepackage{xurl}}{} % add URL line breaks if available
\IfFileExists{bookmark.sty}{\usepackage{bookmark}}{\usepackage{hyperref}}
\hypersetup{
  pdftitle={regis},
  pdfauthor={regis andreoli},
  hidelinks,
  pdfcreator={LaTeX via pandoc}}
\urlstyle{same} % disable monospaced font for URLs
\usepackage[margin=1in]{geometry}
\usepackage{color}
\usepackage{fancyvrb}
\newcommand{\VerbBar}{|}
\newcommand{\VERB}{\Verb[commandchars=\\\{\}]}
\DefineVerbatimEnvironment{Highlighting}{Verbatim}{commandchars=\\\{\}}
% Add ',fontsize=\small' for more characters per line
\usepackage{framed}
\definecolor{shadecolor}{RGB}{248,248,248}
\newenvironment{Shaded}{\begin{snugshade}}{\end{snugshade}}
\newcommand{\AlertTok}[1]{\textcolor[rgb]{0.94,0.16,0.16}{#1}}
\newcommand{\AnnotationTok}[1]{\textcolor[rgb]{0.56,0.35,0.01}{\textbf{\textit{#1}}}}
\newcommand{\AttributeTok}[1]{\textcolor[rgb]{0.77,0.63,0.00}{#1}}
\newcommand{\BaseNTok}[1]{\textcolor[rgb]{0.00,0.00,0.81}{#1}}
\newcommand{\BuiltInTok}[1]{#1}
\newcommand{\CharTok}[1]{\textcolor[rgb]{0.31,0.60,0.02}{#1}}
\newcommand{\CommentTok}[1]{\textcolor[rgb]{0.56,0.35,0.01}{\textit{#1}}}
\newcommand{\CommentVarTok}[1]{\textcolor[rgb]{0.56,0.35,0.01}{\textbf{\textit{#1}}}}
\newcommand{\ConstantTok}[1]{\textcolor[rgb]{0.00,0.00,0.00}{#1}}
\newcommand{\ControlFlowTok}[1]{\textcolor[rgb]{0.13,0.29,0.53}{\textbf{#1}}}
\newcommand{\DataTypeTok}[1]{\textcolor[rgb]{0.13,0.29,0.53}{#1}}
\newcommand{\DecValTok}[1]{\textcolor[rgb]{0.00,0.00,0.81}{#1}}
\newcommand{\DocumentationTok}[1]{\textcolor[rgb]{0.56,0.35,0.01}{\textbf{\textit{#1}}}}
\newcommand{\ErrorTok}[1]{\textcolor[rgb]{0.64,0.00,0.00}{\textbf{#1}}}
\newcommand{\ExtensionTok}[1]{#1}
\newcommand{\FloatTok}[1]{\textcolor[rgb]{0.00,0.00,0.81}{#1}}
\newcommand{\FunctionTok}[1]{\textcolor[rgb]{0.00,0.00,0.00}{#1}}
\newcommand{\ImportTok}[1]{#1}
\newcommand{\InformationTok}[1]{\textcolor[rgb]{0.56,0.35,0.01}{\textbf{\textit{#1}}}}
\newcommand{\KeywordTok}[1]{\textcolor[rgb]{0.13,0.29,0.53}{\textbf{#1}}}
\newcommand{\NormalTok}[1]{#1}
\newcommand{\OperatorTok}[1]{\textcolor[rgb]{0.81,0.36,0.00}{\textbf{#1}}}
\newcommand{\OtherTok}[1]{\textcolor[rgb]{0.56,0.35,0.01}{#1}}
\newcommand{\PreprocessorTok}[1]{\textcolor[rgb]{0.56,0.35,0.01}{\textit{#1}}}
\newcommand{\RegionMarkerTok}[1]{#1}
\newcommand{\SpecialCharTok}[1]{\textcolor[rgb]{0.00,0.00,0.00}{#1}}
\newcommand{\SpecialStringTok}[1]{\textcolor[rgb]{0.31,0.60,0.02}{#1}}
\newcommand{\StringTok}[1]{\textcolor[rgb]{0.31,0.60,0.02}{#1}}
\newcommand{\VariableTok}[1]{\textcolor[rgb]{0.00,0.00,0.00}{#1}}
\newcommand{\VerbatimStringTok}[1]{\textcolor[rgb]{0.31,0.60,0.02}{#1}}
\newcommand{\WarningTok}[1]{\textcolor[rgb]{0.56,0.35,0.01}{\textbf{\textit{#1}}}}
\usepackage{graphicx}
\makeatletter
\def\maxwidth{\ifdim\Gin@nat@width>\linewidth\linewidth\else\Gin@nat@width\fi}
\def\maxheight{\ifdim\Gin@nat@height>\textheight\textheight\else\Gin@nat@height\fi}
\makeatother
% Scale images if necessary, so that they will not overflow the page
% margins by default, and it is still possible to overwrite the defaults
% using explicit options in \includegraphics[width, height, ...]{}
\setkeys{Gin}{width=\maxwidth,height=\maxheight,keepaspectratio}
% Set default figure placement to htbp
\makeatletter
\def\fps@figure{htbp}
\makeatother
\setlength{\emergencystretch}{3em} % prevent overfull lines
\providecommand{\tightlist}{%
  \setlength{\itemsep}{0pt}\setlength{\parskip}{0pt}}
\setcounter{secnumdepth}{-\maxdimen} % remove section numbering
\ifLuaTeX
  \usepackage{selnolig}  % disable illegal ligatures
\fi

\begin{document}
\maketitle

{
\setcounter{tocdepth}{4}
\tableofcontents
}
\hypertarget{load-library}{%
\section{load library}\label{load-library}}

\begin{verbatim}
## Warning: package 'knitr' was built under R version 4.0.5
\end{verbatim}

\begin{verbatim}
## Warning: package 'rstudioapi' was built under R version 4.0.3
\end{verbatim}

\begin{verbatim}
## Warning: package 'ggplot2' was built under R version 4.0.5
\end{verbatim}

\begin{verbatim}
## Warning: package 'gridExtra' was built under R version 4.0.5
\end{verbatim}

\begin{verbatim}
## Warning: package 'dplyr' was built under R version 4.0.5
\end{verbatim}

\begin{verbatim}
## 
## Attaching package: 'dplyr'
\end{verbatim}

\begin{verbatim}
## The following object is masked from 'package:gridExtra':
## 
##     combine
\end{verbatim}

\begin{verbatim}
## The following objects are masked from 'package:stats':
## 
##     filter, lag
\end{verbatim}

\begin{verbatim}
## The following objects are masked from 'package:base':
## 
##     intersect, setdiff, setequal, union
\end{verbatim}

\begin{verbatim}
## Warning: package 'caret' was built under R version 4.0.5
\end{verbatim}

\begin{verbatim}
## Loading required package: lattice
\end{verbatim}

\begin{verbatim}
## Warning: package 'lattice' was built under R version 4.0.5
\end{verbatim}

\begin{verbatim}
## Warning: package 'GGally' was built under R version 4.0.5
\end{verbatim}

\begin{verbatim}
## Registered S3 method overwritten by 'GGally':
##   method from   
##   +.gg   ggplot2
\end{verbatim}

\begin{verbatim}
## Warning: package 'mgcv' was built under R version 4.0.5
\end{verbatim}

\begin{verbatim}
## Loading required package: nlme
\end{verbatim}

\begin{verbatim}
## Warning: package 'nlme' was built under R version 4.0.5
\end{verbatim}

\begin{verbatim}
## 
## Attaching package: 'nlme'
\end{verbatim}

\begin{verbatim}
## The following object is masked from 'package:dplyr':
## 
##     collapse
\end{verbatim}

\begin{verbatim}
## This is mgcv 1.8-38. For overview type 'help("mgcv-package")'.
\end{verbatim}

\begin{verbatim}
## Warning: package 'lubridate' was built under R version 4.0.5
\end{verbatim}

\begin{verbatim}
## 
## Attaching package: 'lubridate'
\end{verbatim}

\begin{verbatim}
## The following objects are masked from 'package:base':
## 
##     date, intersect, setdiff, union
\end{verbatim}

\hypertarget{load-data}{%
\section{Load Data}\label{load-data}}

\hypertarget{data-preparation}{%
\section{Data preparation}\label{data-preparation}}

\begin{Shaded}
\begin{Highlighting}[]
\NormalTok{df\_weather}\SpecialCharTok{$}\NormalTok{local\_time }\OtherTok{\textless{}{-}} \FunctionTok{as.POSIXct}\NormalTok{(df\_weather}\SpecialCharTok{$}\NormalTok{local\_time,}\AttributeTok{tz=}\StringTok{"GMT"}\NormalTok{,}\AttributeTok{format=}\StringTok{"\%Y{-}\%m{-}\%d \%H:\%M"}\NormalTok{)}
\NormalTok{df\_plantA}\SpecialCharTok{$}\NormalTok{Timestamp }\OtherTok{\textless{}{-}} \FunctionTok{as.POSIXct}\NormalTok{(df\_plantA}\SpecialCharTok{$}\NormalTok{Timestamp,}\AttributeTok{tz=}\StringTok{"GMT"}\NormalTok{,}\AttributeTok{format=}\StringTok{"\%Y{-}\%m{-}\%d \%H:\%M:\%S"}\NormalTok{)}
\NormalTok{df\_plantB}\SpecialCharTok{$}\NormalTok{Timestamp }\OtherTok{\textless{}{-}} \FunctionTok{as.POSIXct}\NormalTok{(df\_plantB}\SpecialCharTok{$}\NormalTok{Timestamp,}\AttributeTok{tz=}\StringTok{"GMT"}\NormalTok{,}\AttributeTok{format=}\StringTok{"\%Y{-}\%m{-}\%d \%H:\%M:\%S"}\NormalTok{)}
\NormalTok{df\_plantC}\SpecialCharTok{$}\NormalTok{Timestamp }\OtherTok{\textless{}{-}} \FunctionTok{as.POSIXct}\NormalTok{(df\_plantC}\SpecialCharTok{$}\NormalTok{Timestamp,}\AttributeTok{tz=}\StringTok{"GMT"}\NormalTok{,}\AttributeTok{format=}\StringTok{"\%Y{-}\%m{-}\%d \%H:\%M:\%S"}\NormalTok{)}
\CommentTok{\#df\_weather \%\textgreater{}\% rename(datetime = local\_time,)}
\end{Highlighting}
\end{Shaded}

\begin{Shaded}
\begin{Highlighting}[]
\NormalTok{df\_plantA\_resample }\OtherTok{\textless{}{-}}\NormalTok{ df\_plantA }\SpecialCharTok{\%\textgreater{}\%}
  \FunctionTok{mutate}\NormalTok{(}\AttributeTok{datetime =} \FunctionTok{floor\_date}\NormalTok{(Timestamp, }\StringTok{"1 hour"}\NormalTok{)) }\SpecialCharTok{\%\textgreater{}\%}
  \FunctionTok{group\_by}\NormalTok{(datetime) }\SpecialCharTok{\%\textgreater{}\%}
  \FunctionTok{summarise}\NormalTok{(}\FunctionTok{across}\NormalTok{(Generation\_kW}\SpecialCharTok{:}\NormalTok{Overall\_Consumption\_Calc\_kW, sum))}

\FunctionTok{str}\NormalTok{(df\_plantA\_resample)}
\end{Highlighting}
\end{Shaded}

\begin{verbatim}
## tibble [8,760 x 5] (S3: tbl_df/tbl/data.frame)
##  $ datetime                   : POSIXct[1:8760], format: "2019-01-01 00:00:00" "2019-01-01 01:00:00" ...
##  $ Generation_kW              : num [1:8760] 0 0 0 0 0 ...
##  $ Grid_Feed.In_kW            : num [1:8760] 0 0 0 0 0 0 0 0 0 0 ...
##  $ Grid_Supply_kW             : num [1:8760] 16.9 16.8 17.5 16.9 17.5 ...
##  $ Overall_Consumption_Calc_kW: num [1:8760] 16.9 16.8 17.5 16.9 17.5 ...
\end{verbatim}

\hypertarget{visual-analysis}{%
\section{Visual analysis}\label{visual-analysis}}

First visual analysis of the data. The Graph is Supported by a GAM
smoother.

\begin{Shaded}
\begin{Highlighting}[]
\FunctionTok{ggplot}\NormalTok{(}\AttributeTok{data =}\NormalTok{ df\_plantA\_resample,}
  \AttributeTok{mapping =} \FunctionTok{aes}\NormalTok{(}\AttributeTok{y =}\NormalTok{ Generation\_kW, }\AttributeTok{x =}\NormalTok{ datetime)) }\SpecialCharTok{+}
  \FunctionTok{geom\_point}\NormalTok{(}\AttributeTok{size =} \DecValTok{1}\NormalTok{, }\AttributeTok{color =} \StringTok{"grey69"}\NormalTok{) }\SpecialCharTok{+}
  \FunctionTok{geom\_smooth}\NormalTok{(}\AttributeTok{method =} \StringTok{"gam"}\NormalTok{, }\AttributeTok{color =} \StringTok{"cornflowerblue"}\NormalTok{)}
\end{Highlighting}
\end{Shaded}

\begin{verbatim}
## `geom_smooth()` using formula 'y ~ s(x, bs = "cs")'
\end{verbatim}

\includegraphics{regis_files/figure-latex/unnamed-chunk-5-1.pdf}

\hypertarget{time-series-analysis---with-auto.arima}{%
\section{Time Series Analysis - with
auto.arima}\label{time-series-analysis---with-auto.arima}}

Convert data into time series object. As the data has an hourly
resolution, the time interval is set to 24, which correspond in this
case to a seasonality of one day.

\begin{Shaded}
\begin{Highlighting}[]
\FunctionTok{library}\NormalTok{(forecast)}
\end{Highlighting}
\end{Shaded}

\begin{verbatim}
## Warning: package 'forecast' was built under R version 4.0.5
\end{verbatim}

\begin{verbatim}
## Registered S3 method overwritten by 'quantmod':
##   method            from
##   as.zoo.data.frame zoo
\end{verbatim}

\begin{verbatim}
## 
## Attaching package: 'forecast'
\end{verbatim}

\begin{verbatim}
## The following object is masked from 'package:nlme':
## 
##     getResponse
\end{verbatim}

\begin{Shaded}
\begin{Highlighting}[]
\NormalTok{ts\_1 }\OtherTok{\textless{}{-}} \FunctionTok{ts}\NormalTok{((df\_plantA\_resample}\SpecialCharTok{$}\NormalTok{Generation\_kW), }\AttributeTok{deltat =} \DecValTok{1}\SpecialCharTok{/}\DecValTok{24}\NormalTok{)}
\NormalTok{train }\OtherTok{\textless{}{-}} \FunctionTok{window}\NormalTok{(ts\_1, }\AttributeTok{start =} \DecValTok{1}\NormalTok{, }\AttributeTok{end =} \DecValTok{250}\NormalTok{)}
\NormalTok{fit }\OtherTok{\textless{}{-}} \FunctionTok{auto.arima}\NormalTok{(train)}
\end{Highlighting}
\end{Shaded}

Plot the time series with the prediction.
\includegraphics{regis_files/figure-latex/unnamed-chunk-7-1.pdf}

Plot the time series again, but with a focus on the prediction area. As
it is to see, the model prediction has some noticable, small variation
in the first five days. Then, it converge to value which is a bit

\begin{Shaded}
\begin{Highlighting}[]
\NormalTok{fc }\OtherTok{\textless{}{-}} \FunctionTok{predict}\NormalTok{(fit, }\AttributeTok{n.ahead =} \DecValTok{115}\SpecialCharTok{*}\DecValTok{24}\NormalTok{)}
\FunctionTok{plot}\NormalTok{(ts\_1, }\AttributeTok{lty=}\DecValTok{3}\NormalTok{, }\AttributeTok{cex =} \FloatTok{0.1}\NormalTok{, }\AttributeTok{xlim=}\FunctionTok{c}\NormalTok{(}\DecValTok{240}\NormalTok{, }\DecValTok{270}\NormalTok{), }\AttributeTok{ylim=}\FunctionTok{c}\NormalTok{(}\DecValTok{25}\NormalTok{, }\DecValTok{125}\NormalTok{))}
\FunctionTok{lines}\NormalTok{(train, }\AttributeTok{lwd=}\DecValTok{1}\NormalTok{)}
\FunctionTok{lines}\NormalTok{(fc}\SpecialCharTok{$}\NormalTok{pred, }\AttributeTok{lwd=}\DecValTok{2}\NormalTok{, }\AttributeTok{col=}\StringTok{"red"}\NormalTok{, }\AttributeTok{cex =} \FloatTok{0.1}\NormalTok{)}
\end{Highlighting}
\end{Shaded}

\includegraphics{regis_files/figure-latex/unnamed-chunk-8-1.pdf}

\begin{Shaded}
\begin{Highlighting}[]
\FunctionTok{mean}\NormalTok{(df\_plantA\_resample}\SpecialCharTok{$}\NormalTok{Generation\_kW[df\_plantA\_resample}\SpecialCharTok{$}\NormalTok{Generation\_kW }\SpecialCharTok{\textgreater{}} \DecValTok{0}\NormalTok{])}
\end{Highlighting}
\end{Shaded}

\begin{verbatim}
## [1] 53.46822
\end{verbatim}

Plot the decomposition of the time series with the function stl().

\begin{Shaded}
\begin{Highlighting}[]
\NormalTok{decomp}\OtherTok{\textless{}{-}}\FunctionTok{stl}\NormalTok{(ts\_1, }\AttributeTok{s.window =} \DecValTok{1}\SpecialCharTok{/}\DecValTok{24}\NormalTok{, }\AttributeTok{t.window =} \DecValTok{365}\NormalTok{)}
\FunctionTok{plot}\NormalTok{(decomp)}
\end{Highlighting}
\end{Shaded}

\includegraphics{regis_files/figure-latex/unnamed-chunk-9-1.pdf}

\hypertarget{second-time-series-analysis---manipulated-data}{%
\section{Second Time Series Analysis - Manipulated
Data}\label{second-time-series-analysis---manipulated-data}}

Some artificial years are added to the time series, to be able to
capture the seasonal effect over the entire year.

\hypertarget{data-preparation-1}{%
\subsection{Data preparation}\label{data-preparation-1}}

\begin{Shaded}
\begin{Highlighting}[]
\NormalTok{df\_plantA\_resample\_2 }\OtherTok{\textless{}{-}}\NormalTok{ df\_plantA }\SpecialCharTok{\%\textgreater{}\%}
  \FunctionTok{mutate}\NormalTok{(}\AttributeTok{datetime =} \FunctionTok{floor\_date}\NormalTok{(Timestamp, }\StringTok{"24 hour"}\NormalTok{)) }\SpecialCharTok{\%\textgreater{}\%}
  \FunctionTok{group\_by}\NormalTok{(datetime) }\SpecialCharTok{\%\textgreater{}\%}
  \FunctionTok{summarise}\NormalTok{(}\FunctionTok{across}\NormalTok{(Generation\_kW}\SpecialCharTok{:}\NormalTok{Overall\_Consumption\_Calc\_kW, sum))}

\NormalTok{ts\_2 }\OtherTok{\textless{}{-}} \FunctionTok{ts}\NormalTok{((df\_plantA\_resample\_2}\SpecialCharTok{$}\NormalTok{Generation\_kW), }\AttributeTok{start =} \FunctionTok{c}\NormalTok{(}\DecValTok{2019}\NormalTok{), }\AttributeTok{deltat =} \DecValTok{1}\SpecialCharTok{/}\DecValTok{365}\NormalTok{)}
\end{Highlighting}
\end{Shaded}

Plot time series of one year with a gam smoother.

\begin{Shaded}
\begin{Highlighting}[]
\FunctionTok{ggplot}\NormalTok{(}\AttributeTok{data =}\NormalTok{ df\_plantA\_resample\_2,}
  \AttributeTok{mapping =} \FunctionTok{aes}\NormalTok{(}\AttributeTok{y =}\NormalTok{ Generation\_kW, }\AttributeTok{x =}\NormalTok{ datetime)) }\SpecialCharTok{+}
  \FunctionTok{geom\_point}\NormalTok{(}\AttributeTok{size =} \DecValTok{1}\NormalTok{, }\AttributeTok{color =} \StringTok{"grey69"}\NormalTok{) }\SpecialCharTok{+}
  \FunctionTok{geom\_smooth}\NormalTok{(}\AttributeTok{method =} \StringTok{"gam"}\NormalTok{, }\AttributeTok{color =} \StringTok{"cornflowerblue"}\NormalTok{)}
\end{Highlighting}
\end{Shaded}

\begin{verbatim}
## `geom_smooth()` using formula 'y ~ s(x, bs = "cs")'
\end{verbatim}

\includegraphics{regis_files/figure-latex/unnamed-chunk-11-1.pdf}

Add random noise to first time series, in order to generate new
artificial years.

\begin{Shaded}
\begin{Highlighting}[]
\NormalTok{ts\_artif\_1 }\OtherTok{\textless{}{-}}\NormalTok{ \{}\FunctionTok{jitter}\NormalTok{(ts\_2, }\AttributeTok{factor=}\DecValTok{500}\NormalTok{, }\AttributeTok{amount =} \ConstantTok{NULL}\NormalTok{)\}}
\NormalTok{ts\_artif\_2 }\OtherTok{\textless{}{-}}\NormalTok{ \{}\FunctionTok{jitter}\NormalTok{(ts\_2, }\AttributeTok{factor=}\DecValTok{500}\NormalTok{, }\AttributeTok{amount =} \ConstantTok{NULL}\NormalTok{)\}}
\FunctionTok{head}\NormalTok{(ts\_2)}
\end{Highlighting}
\end{Shaded}

\begin{verbatim}
## Time Series:
## Start = c(2019, 1) 
## End = c(2019, 6) 
## Frequency = 365 
## [1]  71.260 226.412 193.904 242.044   1.248   4.432
\end{verbatim}

\begin{Shaded}
\begin{Highlighting}[]
\FunctionTok{head}\NormalTok{(ts\_artif\_1)}
\end{Highlighting}
\end{Shaded}

\begin{verbatim}
## Time Series:
## Start = c(2019, 1) 
## End = c(2019, 6) 
## Frequency = 365 
## [1] 118.051444 287.398605 165.942028 341.543229   2.641113 -86.668606
\end{verbatim}

\begin{Shaded}
\begin{Highlighting}[]
\FunctionTok{head}\NormalTok{(ts\_artif\_2)}
\end{Highlighting}
\end{Shaded}

\begin{verbatim}
## Time Series:
## Start = c(2019, 1) 
## End = c(2019, 6) 
## Frequency = 365 
## [1]  67.658389 225.880971 140.546870 219.716740 -90.867181   1.136566
\end{verbatim}

Set negative values to zero, as due to the nature of the data, negative
values cannot occur.

\begin{Shaded}
\begin{Highlighting}[]
\NormalTok{ts\_artif\_1[][ts\_artif\_1[] }\SpecialCharTok{\textless{}} \DecValTok{0}\NormalTok{] }\OtherTok{\textless{}{-}} \DecValTok{0}
\NormalTok{ts\_artif\_2[][ts\_artif\_2[] }\SpecialCharTok{\textless{}} \DecValTok{0}\NormalTok{] }\OtherTok{\textless{}{-}} \DecValTok{0}
\end{Highlighting}
\end{Shaded}

Create time series of several years, starting with real data year and
followed by two artificial years. A TS decomposition is then plotted via
the function stl().

\begin{Shaded}
\begin{Highlighting}[]
\NormalTok{ts\_artificial }\OtherTok{\textless{}{-}} \FunctionTok{ts}\NormalTok{(}\FunctionTok{c}\NormalTok{(ts\_2, ts\_artif\_1, ts\_artif\_2), }\AttributeTok{start =} \FunctionTok{c}\NormalTok{(}\DecValTok{2019}\NormalTok{), }\AttributeTok{deltat =} \DecValTok{1}\SpecialCharTok{/}\DecValTok{365}\NormalTok{)}
\NormalTok{decomp}\OtherTok{\textless{}{-}}\FunctionTok{stl}\NormalTok{(ts\_artificial, }\AttributeTok{s.window =} \DecValTok{365}\NormalTok{)}
\FunctionTok{str}\NormalTok{(ts\_artificial)}
\end{Highlighting}
\end{Shaded}

\begin{verbatim}
##  Time-Series [1:1095] from 2019 to 2022: 71.26 226.41 193.9 242.04 1.25 ...
\end{verbatim}

\begin{Shaded}
\begin{Highlighting}[]
\FunctionTok{plot}\NormalTok{(decomp)}
\end{Highlighting}
\end{Shaded}

\includegraphics{regis_files/figure-latex/unnamed-chunk-14-1.pdf}

Train model with the first two years of the time series. To train the
model, the auto.arima function is made use of.

\begin{Shaded}
\begin{Highlighting}[]
\NormalTok{ts\_artificial\_a }\OtherTok{\textless{}{-}} \FunctionTok{ts}\NormalTok{(}\FunctionTok{c}\NormalTok{(ts\_2, ts\_artif\_1, ts\_artif\_2), }\AttributeTok{start =} \FunctionTok{c}\NormalTok{(}\DecValTok{2019}\NormalTok{), }\AttributeTok{frequency =} \DecValTok{365}\NormalTok{)}

\NormalTok{train\_2 }\OtherTok{\textless{}{-}} \FunctionTok{window}\NormalTok{(ts\_artificial\_a, }\AttributeTok{start =} \FunctionTok{c}\NormalTok{(}\DecValTok{2019}\NormalTok{,}\DecValTok{1}\NormalTok{), }\AttributeTok{end =} \FunctionTok{c}\NormalTok{(}\DecValTok{2020}\NormalTok{,}\DecValTok{365}\NormalTok{))}
\NormalTok{fit\_2 }\OtherTok{\textless{}{-}} \FunctionTok{auto.arima}\NormalTok{(train\_2)}
\end{Highlighting}
\end{Shaded}

\begin{verbatim}
## Warning: The chosen seasonal unit root test encountered an error when testing for the first difference.
## From stl(): series is not periodic or has less than two periods
## 0 seasonal differences will be used. Consider using a different unit root test.
\end{verbatim}

\begin{Shaded}
\begin{Highlighting}[]
\NormalTok{arima\_prediciton }\OtherTok{\textless{}{-}} \FunctionTok{predict}\NormalTok{(fit\_2, }\AttributeTok{n.ahead =} \DecValTok{100}\NormalTok{)}

\FunctionTok{plot}\NormalTok{(ts\_artificial\_a, }\AttributeTok{cex =} \FloatTok{0.1}\NormalTok{)}
\FunctionTok{lines}\NormalTok{(train\_2)}
\FunctionTok{lines}\NormalTok{(arima\_prediciton}\SpecialCharTok{$}\NormalTok{pred, }\AttributeTok{col =} \StringTok{"red"}\NormalTok{)}
\end{Highlighting}
\end{Shaded}

\includegraphics{regis_files/figure-latex/unnamed-chunk-15-1.pdf}

\hypertarget{artificial-time-series---new-attemp-use-of-gam-smoother-as-base-for-an-artificial-year-and-add-normally-distributed-noise.}{%
\section{Artificial time series - new attemp: use of gam smoother as
base for an artificial year and add normally distributed
noise.}\label{artificial-time-series---new-attemp-use-of-gam-smoother-as-base-for-an-artificial-year-and-add-normally-distributed-noise.}}

Create new data frame with indexed data. This is done to avoid date
handling

\begin{Shaded}
\begin{Highlighting}[]
\NormalTok{df\_artif }\OtherTok{\textless{}{-}} \FunctionTok{data.frame}\NormalTok{(}\AttributeTok{time =}\NormalTok{ (}\DecValTok{1}\SpecialCharTok{:}\DecValTok{365}\NormalTok{), }\AttributeTok{generation\_kw =}\NormalTok{ df\_plantA\_resample\_2}\SpecialCharTok{$}\NormalTok{Generation\_kW)}
\end{Highlighting}
\end{Shaded}

Compute a smoother line function with the new data frame

\begin{Shaded}
\begin{Highlighting}[]
\NormalTok{gam\_model }\OtherTok{\textless{}{-}} \FunctionTok{gam}\NormalTok{(generation\_kw }\SpecialCharTok{\textasciitilde{}} \FunctionTok{s}\NormalTok{(time), }\AttributeTok{data =}\NormalTok{ df\_artif)}
\end{Highlighting}
\end{Shaded}

Get the discrete steps from the smoother line by computing prediction in
the desired step interval. The result is plotted to visually control the
results. As shown below, the data frame plot is, at least qualitatively,
very similar to the ggplot gam smoother line. This result will now be
taken as smoothed year.

\begin{Shaded}
\begin{Highlighting}[]
\NormalTok{df\_time }\OtherTok{\textless{}{-}} \FunctionTok{data.frame}\NormalTok{(}\AttributeTok{time =} \FunctionTok{c}\NormalTok{(}\DecValTok{1}\SpecialCharTok{:}\DecValTok{365}\NormalTok{))}
\NormalTok{gam\_prediciton }\OtherTok{\textless{}{-}} \FunctionTok{predict}\NormalTok{(gam\_model, }\AttributeTok{newdata =}\NormalTok{ df\_time)}
\FunctionTok{plot}\NormalTok{(gam\_prediciton, }\AttributeTok{cex =} \FloatTok{0.1}\NormalTok{, )}
\end{Highlighting}
\end{Shaded}

\includegraphics{regis_files/figure-latex/unnamed-chunk-19-1.pdf} After
the creation of an artificial ``smoothed'' year, noise is added. In
order to keep the heteroskedastic behavior of the real data, the noise
is added as a multiplication of a random coefficient. After trying
around, the random coefficient are created with values between 0.2 to 1.
A quick look at the plot show a satisfying result.

Add noise to smoothed base year

\begin{Shaded}
\begin{Highlighting}[]
\FunctionTok{library}\NormalTok{(BBmisc)}
\end{Highlighting}
\end{Shaded}

\begin{verbatim}
## Warning: package 'BBmisc' was built under R version 4.0.5
\end{verbatim}

\begin{verbatim}
## 
## Attaching package: 'BBmisc'
\end{verbatim}

\begin{verbatim}
## The following object is masked from 'package:nlme':
## 
##     collapse
\end{verbatim}

\begin{verbatim}
## The following objects are masked from 'package:dplyr':
## 
##     coalesce, collapse
\end{verbatim}

\begin{verbatim}
## The following object is masked from 'package:base':
## 
##     isFALSE
\end{verbatim}

\begin{Shaded}
\begin{Highlighting}[]
\FunctionTok{library}\NormalTok{(fGarch)}
\end{Highlighting}
\end{Shaded}

\begin{verbatim}
## Warning: package 'fGarch' was built under R version 4.0.5
\end{verbatim}

\begin{verbatim}
## Loading required package: timeDate
\end{verbatim}

\begin{verbatim}
## Warning: package 'timeDate' was built under R version 4.0.3
\end{verbatim}

\begin{verbatim}
## Loading required package: timeSeries
\end{verbatim}

\begin{verbatim}
## Warning: package 'timeSeries' was built under R version 4.0.5
\end{verbatim}

\begin{verbatim}
## Loading required package: fBasics
\end{verbatim}

\begin{verbatim}
## Warning: package 'fBasics' was built under R version 4.0.5
\end{verbatim}

\begin{Shaded}
\begin{Highlighting}[]
\FunctionTok{set.seed}\NormalTok{(}\DecValTok{4}\NormalTok{)}
\NormalTok{random\_coef }\OtherTok{\textless{}{-}} \FunctionTok{rsnorm}\NormalTok{(}\DecValTok{365}\NormalTok{, }\AttributeTok{mean =} \DecValTok{1}\NormalTok{, }\AttributeTok{sd =} \DecValTok{1}\NormalTok{, }\AttributeTok{xi =} \FloatTok{0.1}\NormalTok{)}
\NormalTok{ran\_coef\_norm }\OtherTok{\textless{}{-}} \FunctionTok{normalize}\NormalTok{(random\_coef, }\AttributeTok{method =} \StringTok{"range"}\NormalTok{, }\AttributeTok{range =} \FunctionTok{c}\NormalTok{(}\DecValTok{0}\NormalTok{, }\DecValTok{1}\NormalTok{))}
\CommentTok{\#ran\_coef\_norm \textless{}{-} ran\_coef\_norm * 1.3}
\NormalTok{ran\_coef\_norm\_2 }\OtherTok{\textless{}{-}}\NormalTok{ ran\_coef\_norm }\SpecialCharTok{*} \FunctionTok{runif}\NormalTok{(ran\_coef\_norm, }\AttributeTok{min =} \FloatTok{0.3}\NormalTok{, }\AttributeTok{max =} \FloatTok{1.6}\NormalTok{)}
\NormalTok{gam\_yr\_with\_noise }\OtherTok{\textless{}{-}}\NormalTok{ gam\_prediciton }\SpecialCharTok{*}\NormalTok{ ran\_coef\_norm\_2}

\FunctionTok{par}\NormalTok{(}\AttributeTok{mfrow=}\FunctionTok{c}\NormalTok{(}\DecValTok{2}\NormalTok{,}\DecValTok{2}\NormalTok{))}

\FunctionTok{print}\NormalTok{(}\StringTok{"df\_plantA\_resample\_2$Generation\_kW"}\NormalTok{)}
\end{Highlighting}
\end{Shaded}

\begin{verbatim}
## [1] "df_plantA_resample_2$Generation_kW"
\end{verbatim}

\begin{Shaded}
\begin{Highlighting}[]
\FunctionTok{mean}\NormalTok{(df\_plantA\_resample\_2}\SpecialCharTok{$}\NormalTok{Generation\_kW)}
\end{Highlighting}
\end{Shaded}

\begin{verbatim}
## [1] 684.2468
\end{verbatim}

\begin{Shaded}
\begin{Highlighting}[]
\FunctionTok{median}\NormalTok{(df\_plantA\_resample\_2}\SpecialCharTok{$}\NormalTok{Generation\_kW)}
\end{Highlighting}
\end{Shaded}

\begin{verbatim}
## [1] 643.164
\end{verbatim}

\begin{Shaded}
\begin{Highlighting}[]
\FunctionTok{hist}\NormalTok{(df\_plantA\_resample\_2}\SpecialCharTok{$}\NormalTok{Generation\_kW, }\AttributeTok{breaks =} \DecValTok{11}\NormalTok{)}

\FunctionTok{print}\NormalTok{(}\StringTok{""}\NormalTok{)}
\end{Highlighting}
\end{Shaded}

\begin{verbatim}
## [1] ""
\end{verbatim}

\begin{Shaded}
\begin{Highlighting}[]
\FunctionTok{print}\NormalTok{(}\StringTok{"gam\_prediciton"}\NormalTok{)}
\end{Highlighting}
\end{Shaded}

\begin{verbatim}
## [1] "gam_prediciton"
\end{verbatim}

\begin{Shaded}
\begin{Highlighting}[]
\FunctionTok{mean}\NormalTok{(gam\_prediciton)}
\end{Highlighting}
\end{Shaded}

\begin{verbatim}
## [1] 684.2468
\end{verbatim}

\begin{Shaded}
\begin{Highlighting}[]
\FunctionTok{median}\NormalTok{(gam\_prediciton)}
\end{Highlighting}
\end{Shaded}

\begin{verbatim}
## [1] 746.8579
\end{verbatim}

\begin{Shaded}
\begin{Highlighting}[]
\FunctionTok{hist}\NormalTok{(gam\_prediciton, }\AttributeTok{breaks =} \DecValTok{11}\NormalTok{)}

\FunctionTok{print}\NormalTok{(}\StringTok{""}\NormalTok{)}
\end{Highlighting}
\end{Shaded}

\begin{verbatim}
## [1] ""
\end{verbatim}

\begin{Shaded}
\begin{Highlighting}[]
\FunctionTok{print}\NormalTok{(}\StringTok{"gam\_yr\_with\_noise"}\NormalTok{)}
\end{Highlighting}
\end{Shaded}

\begin{verbatim}
## [1] "gam_yr_with_noise"
\end{verbatim}

\begin{Shaded}
\begin{Highlighting}[]
\FunctionTok{hist}\NormalTok{(ran\_coef\_norm\_2)}
\FunctionTok{mean}\NormalTok{(gam\_yr\_with\_noise)}
\end{Highlighting}
\end{Shaded}

\begin{verbatim}
## [1] 504.5858
\end{verbatim}

\begin{Shaded}
\begin{Highlighting}[]
\FunctionTok{median}\NormalTok{(gam\_yr\_with\_noise)}
\end{Highlighting}
\end{Shaded}

\begin{verbatim}
## [1] 402.1255
\end{verbatim}

\begin{Shaded}
\begin{Highlighting}[]
\FunctionTok{plot}\NormalTok{(gam\_yr\_with\_noise, }\AttributeTok{cex =} \FloatTok{0.2}\NormalTok{)}
\end{Highlighting}
\end{Shaded}

\includegraphics{regis_files/figure-latex/unnamed-chunk-20-1.pdf}

\begin{Shaded}
\begin{Highlighting}[]
\FunctionTok{hist}\NormalTok{(gam\_yr\_with\_noise, }\AttributeTok{breaks =} \DecValTok{11}\NormalTok{)}
\end{Highlighting}
\end{Shaded}

\includegraphics{regis_files/figure-latex/unnamed-chunk-20-2.pdf}

While the artificial year visually looks satisfying, the mean generally
is too low. It is not easy to change the noise parameters in such a way,
that the variance and the mean of the artificial year gets near a
similar value of the real data. Therefore, a function is create which
aims to find the best parameter setting:

First, a list of parameter sets is created. This list is then used in a
function that loops through the list and creates for each set a noised
year. The noised year is then compared with the real year by its mean
and variance. A threshold for mean and variance is set. As soon as the
thresholds are cumulatively fulfilled, the function stops and passed the
noised year as an output of the function.

\begin{Shaded}
\begin{Highlighting}[]
\CommentTok{\#create list of parameter sets}
\FunctionTok{library}\NormalTok{(BBmisc)}
\NormalTok{xi }\OtherTok{\textless{}{-}} \FunctionTok{c}\NormalTok{(}\DecValTok{25}\SpecialCharTok{:}\DecValTok{1}\NormalTok{)}\SpecialCharTok{/}\DecValTok{30}
\NormalTok{min }\OtherTok{\textless{}{-}} \FunctionTok{c}\NormalTok{(}\DecValTok{1}\SpecialCharTok{:}\DecValTok{50}\NormalTok{)}\SpecialCharTok{/}\DecValTok{100}
\NormalTok{max }\OtherTok{\textless{}{-}} \FunctionTok{c}\NormalTok{(}\DecValTok{60}\SpecialCharTok{:}\DecValTok{90}\NormalTok{)}\SpecialCharTok{/}\DecValTok{40}
\NormalTok{u }\OtherTok{\textless{}{-}} \FunctionTok{list}\NormalTok{()}
  \ControlFlowTok{for}\NormalTok{ (i }\ControlFlowTok{in}\NormalTok{ xi) \{}
    \ControlFlowTok{for}\NormalTok{ (n }\ControlFlowTok{in}\NormalTok{ min) \{}
      \ControlFlowTok{for}\NormalTok{ (q }\ControlFlowTok{in}\NormalTok{ max) \{}
\NormalTok{        o }\OtherTok{\textless{}{-}} \FunctionTok{c}\NormalTok{(i, n, q)}
\NormalTok{        u }\OtherTok{\textless{}{-}} \FunctionTok{rbind}\NormalTok{(u,o)}
\NormalTok{      \}}
\NormalTok{    \}}
\NormalTok{  \}}
\end{Highlighting}
\end{Shaded}

\begin{Shaded}
\begin{Highlighting}[]
\NormalTok{get\_artificial\_years }\OtherTok{\textless{}{-}} \ControlFlowTok{function}\NormalTok{(mean\_real\_yr, variance\_real\_yr, gam\_prediciton) \{}
  
\NormalTok{  div\_mean }\OtherTok{\textless{}{-}} \DecValTok{0}
\NormalTok{  div\_var }\OtherTok{\textless{}{-}} \DecValTok{0}
\NormalTok{  i }\OtherTok{=} \DecValTok{1}
\NormalTok{  treshold\_1 }\OtherTok{\textless{}{-}} \ConstantTok{TRUE}
\NormalTok{  treshold\_2 }\OtherTok{\textless{}{-}} \ConstantTok{TRUE}
\NormalTok{  df\_sets }\OtherTok{\textless{}{-}} \FunctionTok{list}\NormalTok{()}
\NormalTok{  m }\OtherTok{\textless{}{-}}\NormalTok{ (}\FunctionTok{length}\NormalTok{(u)}\SpecialCharTok{/}\DecValTok{3} \SpecialCharTok{{-}} \DecValTok{1}\NormalTok{)}
  
  \ControlFlowTok{for}\NormalTok{ (n }\ControlFlowTok{in}\NormalTok{ u) \{}
    \ControlFlowTok{if}\NormalTok{ (i }\SpecialCharTok{==}\NormalTok{ m) \{}
      \ControlFlowTok{break}
\NormalTok{    \}}
\NormalTok{    random\_coef }\OtherTok{\textless{}{-}} \FunctionTok{rsnorm}\NormalTok{(}\DecValTok{365}\NormalTok{, }\AttributeTok{mean =} \DecValTok{1}\NormalTok{, }\AttributeTok{sd =} \DecValTok{1}\NormalTok{, }\AttributeTok{xi =} \FunctionTok{as.numeric}\NormalTok{(u[i,][}\DecValTok{1}\NormalTok{]))}
\NormalTok{    ran\_coef\_norm }\OtherTok{\textless{}{-}} \FunctionTok{normalize}\NormalTok{(random\_coef, }\AttributeTok{method =} \StringTok{"range"}\NormalTok{, }\AttributeTok{range =} \FunctionTok{c}\NormalTok{(}\DecValTok{0}\NormalTok{, }\DecValTok{1}\NormalTok{))}
\NormalTok{    ran\_coef\_norm\_2 }\OtherTok{\textless{}{-}}\NormalTok{ ran\_coef\_norm }\SpecialCharTok{*} \FunctionTok{runif}\NormalTok{(ran\_coef\_norm, }\AttributeTok{min =} \FunctionTok{as.numeric}\NormalTok{(u[i,][}\DecValTok{2}\NormalTok{]), }\AttributeTok{max =} \FunctionTok{as.numeric}\NormalTok{(u[i,][}\DecValTok{3}\NormalTok{]))}

\NormalTok{    yr\_with\_noise }\OtherTok{\textless{}{-}}\NormalTok{ gam\_prediciton }\SpecialCharTok{*}\NormalTok{ ran\_coef\_norm\_2}
\NormalTok{    div\_mean }\OtherTok{\textless{}{-}}\NormalTok{ (mean\_real\_yr }\SpecialCharTok{/} \FunctionTok{mean}\NormalTok{(yr\_with\_noise))}
\NormalTok{    div\_var }\OtherTok{\textless{}{-}}\NormalTok{ (variance\_real\_yr }\SpecialCharTok{/} \FunctionTok{var}\NormalTok{(yr\_with\_noise))}
       
\NormalTok{    treshold\_1 }\OtherTok{\textless{}{-}}\NormalTok{ (div\_mean }\SpecialCharTok{\textless{}} \FloatTok{1.025}  \SpecialCharTok{\&\&}\NormalTok{ div\_mean }\SpecialCharTok{\textgreater{}} \FloatTok{0.95}\NormalTok{)}
\NormalTok{    treshold\_2 }\OtherTok{\textless{}{-}}\NormalTok{ (div\_var }\SpecialCharTok{\textless{}} \FloatTok{1.05} \SpecialCharTok{\&\&}\NormalTok{ div\_var }\SpecialCharTok{\textgreater{}} \FloatTok{0.95}\NormalTok{)}
    
    \ControlFlowTok{if}\NormalTok{ (treshold\_1 }\SpecialCharTok{\&\&}\NormalTok{ treshold\_2) \{}
\NormalTok{      df\_sets }\OtherTok{\textless{}{-}} \FunctionTok{rbind}\NormalTok{(df\_sets, yr\_with\_noise)}
      \FunctionTok{print}\NormalTok{(i)}
      \FunctionTok{print}\NormalTok{(div\_mean)}
      \FunctionTok{print}\NormalTok{(div\_var)}
\NormalTok{    \}}
    \ControlFlowTok{if}\NormalTok{ (i }\SpecialCharTok{\%in\%}\NormalTok{ (}\FunctionTok{c}\NormalTok{(}\DecValTok{1}\SpecialCharTok{:}\DecValTok{1000}\NormalTok{)}\SpecialCharTok{*}\DecValTok{500}\NormalTok{))\{}
      \FunctionTok{print}\NormalTok{(i)}
\NormalTok{    \}}
    
\NormalTok{    i }\OtherTok{\textless{}{-}}\NormalTok{ i }\SpecialCharTok{+} \DecValTok{1}
\NormalTok{  \}}
\NormalTok{  df\_sets}
\NormalTok{\}}
\end{Highlighting}
\end{Shaded}

Call the function to compute about 40'000 simulated years and and save
all one that meet the threshold requirements.

\begin{verbatim}
## [1] 500
## [1] 1000
## [1] 1500
## [1] 2000
## [1] 2500
## [1] 3000
## [1] 3500
## [1] 4000
## [1] 4500
## [1] 5000
## [1] 5500
## [1] 6000
## [1] 6500
## [1] 7000
## [1] 7500
## [1] 8000
## [1] 8500
## [1] 9000
## [1] 9500
## [1] 10000
## [1] 10500
## [1] 11000
## [1] 11500
## [1] 12000
## [1] 12500
## [1] 13000
## [1] 13500
## [1] 14000
## [1] 14500
## [1] 15000
## [1] 15498
## [1] 0.9998943
## [1] 0.9574415
## [1] 15500
## [1] 16000
## [1] 16500
## [1] 17000
## [1] 17500
## [1] 18000
## [1] 18500
## [1] 19000
## [1] 19500
## [1] 20000
## [1] 20500
## [1] 21000
## [1] 21500
## [1] 22000
## [1] 22500
## [1] 23000
## [1] 23500
## [1] 24000
## [1] 24500
## [1] 25000
## [1] 25500
## [1] 26000
## [1] 26500
## [1] 27000
## [1] 27500
## [1] 28000
## [1] 28500
## [1] 29000
## [1] 29500
## [1] 30000
## [1] 30500
## [1] 31000
## [1] 31500
## [1] 32000
## [1] 32421
## [1] 1.018895
## [1] 0.9638508
## [1] 32500
## [1] 33000
## [1] 33500
## [1] 34000
## [1] 34500
## [1] 35000
## [1] 35500
## [1] 35515
## [1] 1.022846
## [1] 0.9602693
## [1] 35637
## [1] 1.019101
## [1] 0.9542123
## [1] 36000
## [1] 36500
## [1] 37000
## [1] 37500
## [1] 38000
## [1] 38500
\end{verbatim}

\begin{Shaded}
\begin{Highlighting}[]
\NormalTok{df\_artif }\OtherTok{\textless{}{-}} \FunctionTok{t}\NormalTok{(noised\_year[}\DecValTok{2}\NormalTok{,])}
\NormalTok{arr }\OtherTok{\textless{}{-}} \FunctionTok{array}\NormalTok{(}\FunctionTok{as.numeric}\NormalTok{(}\FunctionTok{unlist}\NormalTok{(df\_artif)))}
\FunctionTok{print}\NormalTok{(}\StringTok{"Real"}\NormalTok{)}
\end{Highlighting}
\end{Shaded}

\begin{verbatim}
## [1] "Real"
\end{verbatim}

\begin{Shaded}
\begin{Highlighting}[]
\FunctionTok{mean}\NormalTok{(df\_plantA\_resample\_2}\SpecialCharTok{$}\NormalTok{Generation\_kW)}
\end{Highlighting}
\end{Shaded}

\begin{verbatim}
## [1] 684.2468
\end{verbatim}

\begin{Shaded}
\begin{Highlighting}[]
\FunctionTok{var}\NormalTok{(df\_plantA\_resample\_2}\SpecialCharTok{$}\NormalTok{Generation\_kW)}
\end{Highlighting}
\end{Shaded}

\begin{verbatim}
## [1] 232351.3
\end{verbatim}

\begin{Shaded}
\begin{Highlighting}[]
\FunctionTok{print}\NormalTok{(}\StringTok{"Artif"}\NormalTok{)}
\end{Highlighting}
\end{Shaded}

\begin{verbatim}
## [1] "Artif"
\end{verbatim}

\begin{Shaded}
\begin{Highlighting}[]
\FunctionTok{mean}\NormalTok{(arr)}
\end{Highlighting}
\end{Shaded}

\begin{verbatim}
## [1] 671.5576
\end{verbatim}

\begin{Shaded}
\begin{Highlighting}[]
\FunctionTok{var}\NormalTok{(arr)}
\end{Highlighting}
\end{Shaded}

\begin{verbatim}
## [1] 241065.6
\end{verbatim}

\begin{Shaded}
\begin{Highlighting}[]
\FunctionTok{plot}\NormalTok{(arr, }\AttributeTok{cex =}\NormalTok{ .}\DecValTok{2}\NormalTok{)}
\end{Highlighting}
\end{Shaded}

\includegraphics{regis_files/figure-latex/unnamed-chunk-24-1.pdf}

\begin{Shaded}
\begin{Highlighting}[]
\FunctionTok{hist}\NormalTok{(df\_plantA\_resample\_2}\SpecialCharTok{$}\NormalTok{Generation\_kW)}
\end{Highlighting}
\end{Shaded}

\includegraphics{regis_files/figure-latex/unnamed-chunk-24-2.pdf}

\begin{Shaded}
\begin{Highlighting}[]
\FunctionTok{hist}\NormalTok{(arr)}
\end{Highlighting}
\end{Shaded}

\includegraphics{regis_files/figure-latex/unnamed-chunk-24-3.pdf}

\begin{Shaded}
\begin{Highlighting}[]
\NormalTok{ts\_artificial\_2 }\OtherTok{\textless{}{-}} \FunctionTok{ts}\NormalTok{(}\FunctionTok{c}\NormalTok{(noised\_year[}\DecValTok{1}\NormalTok{,], noised\_year[}\DecValTok{2}\NormalTok{,], noised\_year[}\DecValTok{3}\NormalTok{,], noised\_year[}\DecValTok{4}\NormalTok{,]), }\AttributeTok{start =} \FunctionTok{c}\NormalTok{(}\DecValTok{2019}\NormalTok{), }\AttributeTok{deltat =} \DecValTok{1}\SpecialCharTok{/}\DecValTok{365}\NormalTok{)}
\NormalTok{decomp\_2 }\OtherTok{\textless{}{-}} \FunctionTok{stl}\NormalTok{(ts\_artificial\_2, }\AttributeTok{s.window =} \DecValTok{1}\SpecialCharTok{/}\DecValTok{24}\NormalTok{, }\AttributeTok{t.window =} \DecValTok{365}\NormalTok{)}
\FunctionTok{plot}\NormalTok{(decomp\_2)}
\end{Highlighting}
\end{Shaded}

\includegraphics{regis_files/figure-latex/unnamed-chunk-25-1.pdf}

\#\texttt{\{r\}\ ts\_artificial\_3\ \textless{}-\ ts(c(noised\_year{[}1,{]},\ noised\_year{[}2,{]},\ noised\_year{[}3,{]},\ noised\_year{[}4,{]},\ noised\_year{[}5,{]}),\ start\ =\ c(2019),\ deltat\ =\ 1/365)\ train\_2\ \textless{}-\ window(ts\_artificial\_3,\ start\ =\ 2019,\ end\ =\ 2024)\ fit\ \textless{}-\ auto.arima(train\_2)}

Plot the time series with the prediction.
\#\texttt{\{r\}\ fc\ \textless{}-\ predict(fit,\ n.ahead\ =\ 115*24)\ plot(ts\_1,\ lty=3,\ size\ =\ 0.1)\ lines(train,\ lwd=1)\ lines(fc\$pred,\ lwd=2,\ col="red",\ size\ =\ 0.01)\ \#lines(fc\$pred+fc\$se*1.96,\ col="red")\ \#lines(fc\$pred-fc\$se*1.96,\ col="red")}

\end{document}
